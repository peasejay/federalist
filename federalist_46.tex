\chapter[No. 46: The Influence of the State and Federal Governments Compared]{No. 46\\ {\small The Influence of the State and Federal Governments Compared}}
To the People of the State of New York:
\vspace{.25cm}

\textsc{Resuming }the subject of the last paper, I proceed to inquire whether the federal government or the State governments will have the advantage with regard to the predilection and support of the people. Notwithstanding the different modes in which they are appointed, we must consider both of them as substantially dependent on the great body of the citizens of the United States. I assume this position here as it respects the first, reserving the proofs for another place. The federal and State governments are in fact but different agents and trustees of the people, constituted with different powers, and designed for different purposes. The adversaries of the Constitution seem to have lost sight of the people altogether in their reasonings on this subject; and to have viewed these different establishments, not only as mutual rivals and enemies, but as uncontrolled by any common superior in their efforts to usurp the authorities of each other. These gentlemen must here be reminded of their error. They must be told that the ultimate authority, wherever the derivative may be found, resides in the people alone, and that it will not depend merely on the comparative ambition or address of the different governments, whether either, or which of them, will be able to enlarge its sphere of jurisdiction at the expense of the other. Truth, no less than decency, requires that the event in every case should be supposed to depend on the sentiments and sanction of their common constituents.

Many considerations, besides those suggested on a former occasion, seem to place it beyond doubt that the first and most natural attachment of the people will be to the governments of their respective States. Into the administration of these a greater number of individuals will expect to rise. From the gift of these a greater number of offices and emoluments will flow. By the superintending care of these, all the more domestic and personal interests of the people will be regulated and provided for. With the affairs of these, the people will be more familiarly and minutely conversant. And with the members of these, will a greater proportion of the people have the ties of personal acquaintance and friendship, and of family and party attachments; on the side of these, therefore, the popular bias may well be expected most strongly to incline.

Experience speaks the same language in this case. The federal administration, though hitherto very defective in comparison with what may be hoped under a better system, had, during the war, and particularly whilst the independent fund of paper emissions was in credit, an activity and importance as great as it can well have in any future circumstances whatever. It was engaged, too, in a course of measures which had for their object the protection of everything that was dear, and the acquisition of everything that could be desirable to the people at large. It was, nevertheless, invariably found, after the transient enthusiasm for the early Congresses was over, that the attention and attachment of the people were turned anew to their own particular governments; that the federal council was at no time the idol of popular favor; and that opposition to proposed enlargements of its powers and importance was the side usually taken by the men who wished to build their political consequence on the prepossessions of their fellow-citizens.

If, therefore, as has been elsewhere remarked, the people should in future become more partial to the federal than to the State governments, the change can only result from such manifest and irresistible proofs of a better administration, as will overcome all their antecedent propensities. And in that case, the people ought not surely to be precluded from giving most of their confidence where they may discover it to be most due; but even in that case the State governments could have little to apprehend, because it is only within a certain sphere that the federal power can, in the nature of things, be advantageously administered.

The remaining points on which I propose to compare the federal and State governments, are the disposition and the faculty they may respectively possess, to resist and frustrate the measures of each other.

It has been already proved that the members of the federal will be more dependent on the members of the State governments, than the latter will be on the former. It has appeared also, that the prepossessions of the people, on whom both will depend, will be more on the side of the State governments, than of the federal government. So far as the disposition of each towards the other may be influenced by these causes, the State governments must clearly have the advantage. But in a distinct and very important point of view, the advantage will lie on the same side. The prepossessions, which the members themselves will carry into the federal government, will generally be favorable to the States; whilst it will rarely happen, that the members of the State governments will carry into the public councils a bias in favor of the general government. A local spirit will infallibly prevail much more in the members of Congress, than a national spirit will prevail in the legislatures of the particular States. Every one knows that a great proportion of the errors committed by the State legislatures proceeds from the disposition of the members to sacrifice the comprehensive and permanent interest of the State, to the particular and separate views of the counties or districts in which they reside. And if they do not sufficiently enlarge their policy to embrace the collective welfare of their particular State, how can it be imagined that they will make the aggregate prosperity of the Union, and the dignity and respectability of its government, the objects of their affections and consultations? For the same reason that the members of the State legislatures will be unlikely to attach themselves sufficiently to national objects, the members of the federal legislature will be likely to attach themselves too much to local objects. The States will be to the latter what counties and towns are to the former. Measures will too often be decided according to their probable effect, not on the national prosperity and happiness, but on the prejudices, interests, and pursuits of the governments and people of the individual States. What is the spirit that has in general characterized the proceedings of Congress? A perusal of their journals, as well as the candid acknowledgments of such as have had a seat in that assembly, will inform us, that the members have but too frequently displayed the character, rather of partisans of their respective States, than of impartial guardians of a common interest; that where on one occasion improper sacrifices have been made of local considerations, to the aggrandizement of the federal government, the great interests of the nation have suffered on a hundred, from an undue attention to the local prejudices, interests, and views of the particular States. I mean not by these reflections to insinuate, that the new federal government will not embrace a more enlarged plan of policy than the existing government may have pursued; much less, that its views will be as confined as those of the State legislatures; but only that it will partake sufficiently of the spirit of both, to be disinclined to invade the rights of the individual States, or the prerogatives of their governments. The motives on the part of the State governments, to augment their prerogatives by defalcations from the federal government, will be overruled by no reciprocal predispositions in the members.

Were it admitted, however, that the Federal government may feel an equal disposition with the State governments to extend its power beyond the due limits, the latter would still have the advantage in the means of defeating such encroachments. If an act of a particular State, though unfriendly to the national government, be generally popular in that State and should not too grossly violate the oaths of the State officers, it is executed immediately and, of course, by means on the spot and depending on the State alone. The opposition of the federal government, or the interposition of federal officers, would but inflame the zeal of all parties on the side of the State, and the evil could not be prevented or repaired, if at all, without the employment of means which must always be resorted to with reluctance and difficulty. On the other hand, should an unwarrantable measure of the federal government be unpopular in particular States, which would seldom fail to be the case, or even a warrantable measure be so, which may sometimes be the case, the means of opposition to it are powerful and at hand. The disquietude of the people; their repugnance and, perhaps, refusal to co-operate with the officers of the Union; the frowns of the executive magistracy of the State; the embarrassments created by legislative devices, which would often be added on such occasions, would oppose, in any State, difficulties not to be despised; would form, in a large State, very serious impediments; and where the sentiments of several adjoining States happened to be in unison, would present obstructions which the federal government would hardly be willing to encounter.

But ambitious encroachments of the federal government, on the authority of the State governments, would not excite the opposition of a single State, or of a few States only. They would be signals of general alarm. Every government would espouse the common cause. A correspondence would be opened. Plans of resistance would be concerted. One spirit would animate and conduct the whole. The same combinations, in short, would result from an apprehension of the federal, as was produced by the dread of a foreign, yoke; and unless the projected innovations should be voluntarily renounced, the same appeal to a trial of force would be made in the one case as was made in the other. But what degree of madness could ever drive the federal government to such an extremity. In the contest with Great Britain, one part of the empire was employed against the other. The more numerous part invaded the rights of the less numerous part. The attempt was unjust and unwise; but it was not in speculation absolutely chimerical. But what would be the contest in the case we are supposing? Who would be the parties? A few representatives of the people would be opposed to the people themselves; or rather one set of representatives would be contending against thirteen sets of representatives, with the whole body of their common constituents on the side of the latter.

The only refuge left for those who prophesy the downfall of the State governments is the visionary supposition that the federal government may previously accumulate a military force for the projects of ambition. The reasonings contained in these papers must have been employed to little purpose indeed, if it could be necessary now to disprove the reality of this danger. That the people and the States should, for a sufficient period of time, elect an uninterrupted succession of men ready to betray both; that the traitors should, throughout this period, uniformly and systematically pursue some fixed plan for the extension of the military establishment; that the governments and the people of the States should silently and patiently behold the gathering storm, and continue to supply the materials, until it should be prepared to burst on their own heads, must appear to every one more like the incoherent dreams of a delirious jealousy, or the misjudged exaggerations of a counterfeit zeal, than like the sober apprehensions of genuine patriotism. Extravagant as the supposition is, let it however be made. Let a regular army, fully equal to the resources of the country, be formed; and let it be entirely at the devotion of the federal government; still it would not be going too far to say, that the State governments, with the people on their side, would be able to repel the danger. The highest number to which, according to the best computation, a standing army can be carried in any country, does not exceed one hundredth part of the whole number of souls; or one twenty-fifth part of the number able to bear arms. This proportion would not yield, in the United States, an army of more than twenty-five or thirty thousand men. To these would be opposed a militia amounting to near half a million of citizens with arms in their hands, officered by men chosen from among themselves, fighting for their common liberties, and united and conducted by governments possessing their affections and confidence. It may well be doubted, whether a militia thus circumstanced could ever be conquered by such a proportion of regular troops. Those who are best acquainted with the last successful resistance of this country against the British arms, will be most inclined to deny the possibility of it. Besides the advantage of being armed, which the Americans possess over the people of almost every other nation, the existence of subordinate governments, to which the people are attached, and by which the militia officers are appointed, forms a barrier against the enterprises of ambition, more insurmountable than any which a simple government of any form can admit of. Notwithstanding the military establishments in the several kingdoms of Europe, which are carried as far as the public resources will bear, the governments are afraid to trust the people with arms. And it is not certain, that with this aid alone they would not be able to shake off their yokes. But were the people to possess the additional advantages of local governments chosen by themselves, who could collect the national will and direct the national force, and of officers appointed out of the militia, by these governments, and attached both to them and to the militia, it may be affirmed with the greatest assurance, that the throne of every tyranny in Europe would be speedily overturned in spite of the legions which surround it. Let us not insult the free and gallant citizens of America with the suspicion, that they would be less able to defend the rights of which they would be in actual possession, than the debased subjects of arbitrary power would be to rescue theirs from the hands of their oppressors. Let us rather no longer insult them with the supposition that they can ever reduce themselves to the necessity of making the experiment, by a blind and tame submission to the long train of insidious measures which must precede and produce it.

The argument under the present head may be put into a very concise form, which appears altogether conclusive. Either the mode in which the federal government is to be constructed will render it sufficiently dependent on the people, or it will not. On the first supposition, it will be restrained by that dependence from forming schemes obnoxious to their constituents. On the other supposition, it will not possess the confidence of the people, and its schemes of usurpation will be easily defeated by the State governments, who will be supported by the people.

On summing up the considerations stated in this and the last paper, they seem to amount to the most convincing evidence, that the powers proposed to be lodged in the federal government are as little formidable to those reserved to the individual States, as they are indispensably necessary to accomplish the purposes of the Union; and that all those alarms which have been sounded, of a meditated and consequential annihilation of the State governments, must, on the most favorable interpretation, be ascribed to the chimerical fears of the authors of them.

\vspace{.5cm}
\textsc{Publius}

\chapter[No. 58: Objection That The Number of Members Will Not Be Augmented as the Progress of Population Demands.]{No. 58\\ {\small Objection That The Number of Members Will Not Be Augmented as the Progress of Population Demands.}}
To the People of the State of New York:
\vspace{.4cm}

\textsc{The} remaining charge against the House of Representatives, which I am to examine, is grounded on a supposition that the number of members will not be augmented from time to time, as the progress of population may demand.

It has been admitted, that this objection, if well supported, would have great weight. The following observations will show that, like most other objections against the Constitution, it can only proceed from a partial view of the subject, or from a jealousy which discolors and disfigures every object which is beheld.

Within every successive term of ten years a census of inhabitants is to be repeated. The unequivocal objects of these regulations are, first, to readjust, from time to time, the apportionment of representatives to the number of inhabitants, under the single exception that each State shall have one representative at least; secondly, to augment the number of representatives at the same periods, under the sole limitation that the whole number shall not exceed one for every thirty thousand inhabitants. If we review the constitutions of the several States, we shall find that some of them contain no determinate regulations on this subject, that others correspond pretty much on this point with the federal Constitution, and that the most effectual security in any of them is resolvable into a mere directory provision.

It may be alleged, perhaps, that the Senate would be prompted by like motives to an adverse coalition; and as their concurrence would be indispensable, the just and constitutional views of the other branch might be defeated. This is the difficulty which has probably created the most serious apprehensions in the jealous friends of a numerous representation. Fortunately it is among the difficulties which, existing only in appearance, vanish on a close and accurate inspection. The following reflections will, if I mistake not, be admitted to be conclusive and satisfactory on this point.

Notwithstanding the equal authority which will subsist between the two houses on all legislative subjects, except the originating of money bills, it cannot be doubted that the House, composed of the greater number of members, when supported by the more powerful States, and speaking the known and determined sense of a majority of the people, will have no small advantage in a question depending on the comparative firmness of the two houses.

This advantage must be increased by the consciousness, felt by the same side of being supported in its demands by right, by reason, and by the Constitution; and the consciousness, on the opposite side, of contending against the force of all these solemn considerations.

It is farther to be considered, that in the gradation between the smallest and largest States, there are several, which, though most likely in general to arrange themselves among the former are too little removed in extent and population from the latter, to second an opposition to their just and legitimate pretensions. Hence it is by no means certain that a majority of votes, even in the Senate, would be unfriendly to proper augmentations in the number of representatives.

It will not be looking too far to add, that the senators from all the new States may be gained over to the just views of the House of Representatives, by an expedient too obvious to be overlooked. As these States will, for a great length of time, advance in population with peculiar rapidity, they will be interested in frequent reapportionments of the representatives to the number of inhabitants. The large States, therefore, who will prevail in the House of Representatives, will have nothing to do but to make reapportionments and augmentations mutually conditions of each other; and the senators from all the most growing States will be bound to contend for the latter, by the interest which their States will feel in the former.

These considerations seem to afford ample security on this subject, and ought alone to satisfy all the doubts and fears which have been indulged with regard to it. Admitting, however, that they should all be insufficient to subdue the unjust policy of the smaller States, or their predominant influence in the councils of the Senate, a constitutional and infallible resource still remains with the larger States, by which they will be able at all times to accomplish their just purposes. The House of Representatives cannot only refuse, but they alone can propose, the supplies requisite for the support of government. They, in a word, hold the purse--that powerful instrument by which we behold, in the history of the British Constitution, an infant and humble representation of the people gradually enlarging the sphere of its activity and importance, and finally reducing, as far as it seems to have wished, all the overgrown prerogatives of the other branches of the government. This power over the purse may, in fact, be regarded as the most complete and effectual weapon with which any constitution can arm the immediate representatives of the people, for obtaining a redress of every grievance, and for carrying into effect every just and salutary measure.

But will not the House of Representatives be as much interested as the Senate in maintaining the government in its proper functions, and will they not therefore be unwilling to stake its existence or its reputation on the pliancy of the Senate? Or, if such a trial of firmness between the two branches were hazarded, would not the one be as likely first to yield as the other? These questions will create no difficulty with those who reflect that in all cases the smaller the number, and the more permanent and conspicuous the station, of men in power, the stronger must be the interest which they will individually feel in whatever concerns the government. Those who represent the dignity of their country in the eyes of other nations, will be particularly sensible to every prospect of public danger, or of dishonorable stagnation in public affairs. To those causes we are to ascribe the continual triumph of the British House of Commons over the other branches of the government, whenever the engine of a money bill has been employed. An absolute inflexibility on the side of the latter, although it could not have failed to involve every department of the state in the general confusion, has neither been apprehended nor experienced. The utmost degree of firmness that can be displayed by the federal Senate or President, will not be more than equal to a resistance in which they will be supported by constitutional and patriotic principles.

In this review of the Constitution of the House of Representatives, I have passed over the circumstances of economy, which, in the present state of affairs, might have had some effect in lessening the temporary number of representatives, and a disregard of which would probably have been as rich a theme of declamation against the Constitution as has been shown by the smallness of the number proposed. I omit also any remarks on the difficulty which might be found, under present circumstances, in engaging in the federal service a large number of such characters as the people will probably elect. One observation, however, I must be permitted to add on this subject as claiming, in my judgment, a very serious attention. It is, that in all legislative assemblies the greater the number composing them may be, the fewer will be the men who will in fact direct their proceedings. In the first place, the more numerous an assembly may be, of whatever characters composed, the greater is known to be the ascendency of passion over reason. In the next place, the larger the number, the greater will be the proportion of members of limited information and of weak capacities. Now, it is precisely on characters of this description that the eloquence and address of the few are known to act with all their force. In the ancient republics, where the whole body of the people assembled in person, a single orator, or an artful statesman, was generally seen to rule with as complete a sway as if a sceptre had been placed in his single hand. On the same principle, the more multitudinous a representative assembly may be rendered, the more it will partake of the infirmities incident to collective meetings of the people. Ignorance will be the dupe of cunning, and passion the slave of sophistry and declamation. The people can never err more than in supposing that by multiplying their representatives beyond a certain limit, they strengthen the barrier against the government of a few. Experience will forever admonish them that, on the contrary, \textsc{after securing a sufficient number for the purposes of safety}, \textsc{of local information}, \textsc{and of diffusive sympathy with the whole society}, they will counteract their own views by every addition to their representatives. The countenance of the government may become more democratic, but the soul that animates it will be more oligarchic. The machine will be enlarged, but the fewer, and often the more secret, will be the springs by which its motions are directed.

As connected with the objection against the number of representatives, may properly be here noticed, that which has been suggested against the number made competent for legislative business. It has been said that more than a majority ought to have been required for a quorum; and in particular cases, if not in all, more than a majority of a quorum for a decision. That some advantages might have resulted from such a precaution, cannot be denied. It might have been an additional shield to some particular interests, and another obstacle generally to hasty and partial measures. But these considerations are outweighed by the inconveniences in the opposite scale. In all cases where justice or the general good might require new laws to be passed, or active measures to be pursued, the fundamental principle of free government would be reversed. It would be no longer the majority that would rule: the power would be transferred to the minority. Were the defensive privilege limited to particular cases, an interested minority might take advantage of it to screen themselves from equitable sacrifices to the general weal, or, in particular emergencies, to extort unreasonable indulgences. Lastly, it would facilitate and foster the baneful practice of secessions; a practice which has shown itself even in States where a majority only is required; a practice subversive of all the principles of order and regular government; a practice which leads more directly to public convulsions, and the ruin of popular governments, than any other which has yet been displayed among us.

\vspace{.5cm}
\textsc{Publius}

\vspace{1.5cm}


\chapter[No. 9: The Union as a Safeguard Against Domestic Faction and Insurrection]{No. 9\\ {\small The Union as a Safeguard Against Domestic Faction and Insurrection}}
To the People of the State of New York:
\vspace{.4cm}

\textsc{A firm u}nion will be of the utmost moment to the peace and liberty of the States, as a barrier against domestic faction and insurrection. It is impossible to read the history of the petty republics of Greece and Italy without feeling sensations of horror and disgust at the distractions with which they were continually agitated, and at the rapid succession of revolutions by which they were kept in a state of perpetual vibration between the extremes of tyranny and anarchy. If they exhibit occasional calms, these only serve as short-lived contrast to the furious storms that are to succeed. If now and then intervals of felicity open to view, we behold them with a mixture of regret, arising from the reflection that the pleasing scenes before us are soon to be overwhelmed by the tempestuous waves of sedition and party rage. If momentary rays of glory break forth from the gloom, while they dazzle us with a transient and fleeting brilliancy, they at the same time admonish us to lament that the vices of government should pervert the direction and tarnish the lustre of those bright talents and exalted endowments for which the favored soils that produced them have been so justly celebrated.

From the disorders that disfigure the annals of those republics the advocates of despotism have drawn arguments, not only against the forms of republican government, but against the very principles of civil liberty. They have decried all free government as inconsistent with the order of society, and have indulged themselves in malicious exultation over its friends and partisans. Happily for mankind, stupendous fabrics reared on the basis of liberty, which have flourished for ages, have, in a few glorious instances, refuted their gloomy sophisms. And, I trust, America will be the broad and solid foundation of other edifices, not less magnificent, which will be equally permanent monuments of their errors.

But it is not to be denied that the portraits they have sketched of republican government were too just copies of the originals from which they were taken. If it had been found impracticable to have devised models of a more perfect structure, the enlightened friends to liberty would have been obliged to abandon the cause of that species of government as indefensible. The science of politics, however, like most other sciences, has received great improvement. The efficacy of various principles is now well understood, which were either not known at all, or imperfectly known to the ancients. The regular distribution of power into distinct departments; the introduction of legislative balances and checks; the institution of courts composed of judges holding their offices during good behavior; the representation of the people in the legislature by deputies of their own election: these are wholly new discoveries, or have made their principal progress towards perfection in modern times. They are means, and powerful means, by which the excellences of republican government may be retained and its imperfections lessened or avoided. To this catalogue of circumstances that tend to the amelioration of popular systems of civil government, I shall venture, however novel it may appear to some, to add one more, on a principle which has been made the foundation of an objection to the new Constitution; I mean the \textsc{enlargement} of the \textsc{orbit} within which such systems are to revolve, either in respect to the dimensions of a single State or to the consolidation of several smaller States into one great Confederacy. The latter is that which immediately concerns the object under consideration. It will, however, be of use to examine the principle in its application to a single State, which shall be attended to in another place.

The utility of a Confederacy, as well to suppress faction and to guard the internal tranquillity of States, as to increase their external force and security, is in reality not a new idea. It has been practiced upon in different countries and ages, and has received the sanction of the most approved writers on the subject of politics. The opponents of the plan proposed have, with great assiduity, cited and circulated the observations of Montesquieu on the necessity of a contracted territory for a republican government. But they seem not to have been apprised of the sentiments of that great man expressed in another part of his work, nor to have adverted to the consequences of the principle to which they subscribe with such ready acquiescence.

When Montesquieu recommends a small extent for republics, the standards he had in view were of dimensions far short of the limits of almost every one of these States. Neither Virginia, Massachusetts, Pennsylvania, New York, North Carolina, nor Georgia can by any means be compared with the models from which he reasoned and to which the terms of his description apply. If we therefore take his ideas on this point as the criterion of truth, we shall be driven to the alternative either of taking refuge at once in the arms of monarchy, or of splitting ourselves into an infinity of little, jealous, clashing, tumultuous commonwealths, the wretched nurseries of unceasing discord, and the miserable objects of universal pity or contempt. Some of the writers who have come forward on the other side of the question seem to have been aware of the dilemma; and have even been bold enough to hint at the division of the larger States as a desirable thing. Such an infatuated policy, such a desperate expedient, might, by the multiplication of petty offices, answer the views of men who possess not qualifications to extend their influence beyond the narrow circles of personal intrigue, but it could never promote the greatness or happiness of the people of America.

Referring the examination of the principle itself to another place, as has been already mentioned, it will be sufficient to remark here that, in the sense of the author who has been most emphatically quoted upon the occasion, it would only dictate a reduction of the \textsc{size} of the more considerable \textsc{members} of the Union, but would not militate against their being all comprehended in one confederate government. And this is the true question, in the discussion of which we are at present interested.

So far are the suggestions of Montesquieu from standing in opposition to a general Union of the States, that he explicitly treats of a confederate republic as the expedient for extending the sphere of popular government, and reconciling the advantages of monarchy with those of republicanism.

``It is very probable," (says he\footnote{``Spirit of Laws," vol. i., book ix., chap. i.}) ``that mankind would have been obliged at length to live constantly under the government of a single person, had they not contrived a kind of constitution that has all the internal advantages of a republican, together with the external force of a monarchical government. I mean a \textsc{confederate republic}."

``This form of government is a convention by which several smaller \textsc{states} agree to become members of a larger ONE, which they intend to form. It is a kind of assemblage of societies that constitute a new one, capable of increasing, by means of new associations, till they arrive to such a degree of power as to be able to provide for the security of the united body."

``A republic of this kind, able to withstand an external force, may support itself without any internal corruptions. The form of this society prevents all manner of inconveniences."

``If a single member should attempt to usurp the supreme authority, he could not be supposed to have an equal authority and credit in all the confederate states. Were he to have too great influence over one, this would alarm the rest. Were he to subdue a part, that which would still remain free might oppose him with forces independent of those which he had usurped and overpower him before he could be settled in his usurpation."

``Should a popular insurrection happen in one of the confederate states the others are able to quell it. Should abuses creep into one part, they are reformed by those that remain sound. The state may be destroyed on one side, and not on the other; the confederacy may be dissolved, and the confederates preserve their sovereignty."

``As this government is composed of small republics, it enjoys the internal happiness of each; and with respect to its external situation, it is possessed, by means of the association, of all the advantages of large monarchies."

I have thought it proper to quote at length these interesting passages, because they contain a luminous abridgment of the principal arguments in favor of the Union, and must effectually remove the false impressions which a misapplication of other parts of the work was calculated to make. They have, at the same time, an intimate connection with the more immediate design of this paper; which is, to illustrate the tendency of the Union to repress domestic faction and insurrection.

A distinction, more subtle than accurate, has been raised between a \textsc{confederacy} and a \textsc{consolidation} of the States. The essential characteristic of the first is said to be, the restriction of its authority to the members in their collective capacities, without reaching to the individuals of whom they are composed. It is contended that the national council ought to have no concern with any object of internal administration. An exact equality of suffrage between the members has also been insisted upon as a leading feature of a confederate government. These positions are, in the main, arbitrary; they are supported neither by principle nor precedent. It has indeed happened, that governments of this kind have generally operated in the manner which the distinction taken notice of, supposes to be inherent in their nature; but there have been in most of them extensive exceptions to the practice, which serve to prove, as far as example will go, that there is no absolute rule on the subject. And it will be clearly shown in the course of this investigation that as far as the principle contended for has prevailed, it has been the cause of incurable disorder and imbecility in the government.

The definition of a \textsc{confederate republic} seems simply to be ``an assemblage of societies," or an association of two or more states into one state. The extent, modifications, and objects of the federal authority are mere matters of discretion. So long as the separate organization of the members be not abolished; so long as it exists, by a constitutional necessity, for local purposes; though it should be in perfect subordination to the general authority of the union, it would still be, in fact and in theory, an association of states, or a confederacy. The proposed Constitution, so far from implying an abolition of the State governments, makes them constituent parts of the national sovereignty, by allowing them a direct representation in the Senate, and leaves in their possession certain exclusive and very important portions of sovereign power. This fully corresponds, in every rational import of the terms, with the idea of a federal government.

In the Lycian confederacy, which consisted of twenty-three \textsc{cities} or republics, the largest were entitled to \textsc{three} votes in the \textsc{common council}, those of the middle class to TWO, and the smallest to ONE. The \textsc{common council} had the appointment of all the judges and magistrates of the respective \textsc{cities}. This was certainly the most, delicate species of interference in their internal administration; for if there be any thing that seems exclusively appropriated to the local jurisdictions, it is the appointment of their own officers. Yet Montesquieu, speaking of this association, says: ``Were I to give a model of an excellent Confederate Republic, it would be that of Lycia." Thus we perceive that the distinctions insisted upon were not within the contemplation of this enlightened civilian; and we shall be led to conclude, that they are the novel refinements of an erroneous theory.

\vspace{.5cm}
\textsc{Publius}

\vspace{1.5cm}


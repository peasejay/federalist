\chapter[No. 83: The Judiciary Continued in Relation to Trial by Jury]{No. 83\\ {\small The Judiciary Continued in Relation to Trial by Jury}}
To the People of the State of New York:
\vspace{.4cm}

\textsc{The }objection to the plan of the convention, which has met with most success in this State, and perhaps in several of the other States, is that relative to the want of a constitutional provision for the trial by jury in civil cases. The disingenuous form in which this objection is usually stated has been repeatedly adverted to and exposed, but continues to be pursued in all the conversations and writings of the opponents of the plan. The mere silence of the Constitution in regard to civil causes, is represented as an abolition of the trial by jury, and the declamations to which it has afforded a pretext are artfully calculated to induce a persuasion that this pretended abolition is complete and universal, extending not only to every species of civil, but even to criminal causes. To argue with respect to the latter would, however, be as vain and fruitless as to attempt the serious proof of the existence of matter, or to demonstrate any of those propositions which, by their own internal evidence, force conviction, when expressed in language adapted to convey their meaning.

With regard to civil causes, subtleties almost too contemptible for refutation have been employed to countenance the surmise that a thing which is only not provided for, is entirely abolished. Every man of discernment must at once perceive the wide difference between silence and abolition. But as the inventors of this fallacy have attempted to support it by certain legal maxims of interpretation, which they have perverted from their true meaning, it may not be wholly useless to explore the ground they have taken.

The maxims on which they rely are of this nature: ``A specification of particulars is an exclusion of generals"; or, ``The expression of one thing is the exclusion of another." Hence, say they, as the Constitution has established the trial by jury in criminal cases, and is silent in respect to civil, this silence is an implied prohibition of trial by jury in regard to the latter.

The rules of legal interpretation are rules of common sense, adopted by the courts in the construction of the laws. The true test, therefore, of a just application of them is its conformity to the source from which they are derived. This being the case, let me ask if it is consistent with common-sense to suppose that a provision obliging the legislative power to commit the trial of criminal causes to juries, is a privation of its right to authorize or permit that mode of trial in other cases? Is it natural to suppose, that a command to do one thing is a prohibition to the doing of another, which there was a previous power to do, and which is not incompatible with the thing commanded to be done? If such a supposition would be unnatural and unreasonable, it cannot be rational to maintain that an injunction of the trial by jury in certain cases is an interdiction of it in others.

A power to constitute courts is a power to prescribe the mode of trial; and consequently, if nothing was said in the Constitution on the subject of juries, the legislature would be at liberty either to adopt that institution or to let it alone. This discretion, in regard to criminal causes, is abridged by the express injunction of trial by jury in all such cases; but it is, of course, left at large in relation to civil causes, there being a total silence on this head. The specification of an obligation to try all criminal causes in a particular mode, excludes indeed the obligation or necessity of employing the same mode in civil causes, but does not abridge the power of the legislature to exercise that mode if it should be thought proper. The pretense, therefore, that the national legislature would not be at full liberty to submit all the civil causes of federal cognizance to the determination of juries, is a pretense destitute of all just foundation.

From these observations this conclusion results: that the trial by jury in civil cases would not be abolished; and that the use attempted to be made of the maxims which have been quoted, is contrary to reason and common-sense, and therefore not admissible. Even if these maxims had a precise technical sense, corresponding with the idea of those who employ them upon the present occasion, which, however, is not the case, they would still be inapplicable to a constitution of government. In relation to such a subject, the natural and obvious sense of its provisions, apart from any technical rules, is the true criterion of construction.

Having now seen that the maxims relied upon will not bear the use made of them, let us endeavor to ascertain their proper use and true meaning. This will be best done by examples. The plan of the convention declares that the power of Congress, or, in other words, of the national legislature, shall extend to certain enumerated cases. This specification of particulars evidently excludes all pretension to a general legislative authority, because an affirmative grant of special powers would be absurd, as well as useless, if a general authority was intended.

In like manner the judicial authority of the federal judicatures is declared by the Constitution to comprehend certain cases particularly specified. The expression of those cases marks the precise limits, beyond which the federal courts cannot extend their jurisdiction, because the objects of their cognizance being enumerated, the specification would be nugatory if it did not exclude all ideas of more extensive authority.

These examples are sufficient to elucidate the maxims which have been mentioned, and to designate the manner in which they should be used. But that there may be no misapprehensions upon this subject, I shall add one case more, to demonstrate the proper use of these maxims, and the abuse which has been made of them.

Let us suppose that by the laws of this State a married woman was incapable of conveying her estate, and that the legislature, considering this as an evil, should enact that she might dispose of her property by deed executed in the presence of a magistrate. In such a case there can be no doubt but the specification would amount to an exclusion of any other mode of conveyance, because the woman having no previous power to alienate her property, the specification determines the particular mode which she is, for that purpose, to avail herself of. But let us further suppose that in a subsequent part of the same act it should be declared that no woman should dispose of any estate of a determinate value without the consent of three of her nearest relations, signified by their signing the deed; could it be inferred from this regulation that a married woman might not procure the approbation of her relations to a deed for conveying property of inferior value? The position is too absurd to merit a refutation, and yet this is precisely the position which those must establish who contend that the trial by juries in civil cases is abolished, because it is expressly provided for in cases of a criminal nature.

From these observations it must appear unquestionably true, that trial by jury is in no case abolished by the proposed Constitution, and it is equally true, that in those controversies between individuals in which the great body of the people are likely to be interested, that institution will remain precisely in the same situation in which it is placed by the State constitutions, and will be in no degree altered or influenced by the adoption of the plan under consideration. The foundation of this assertion is, that the national judiciary will have no cognizance of them, and of course they will remain determinable as heretofore by the State courts only, and in the manner which the State constitutions and laws prescribe. All land causes, except where claims under the grants of different States come into question, and all other controversies between the citizens of the same State, unless where they depend upon positive violations of the articles of union, by acts of the State legislatures, will belong exclusively to the jurisdiction of the State tribunals. Add to this, that admiralty causes, and almost all those which are of equity jurisdiction, are determinable under our own government without the intervention of a jury, and the inference from the whole will be, that this institution, as it exists with us at present, cannot possibly be affected to any great extent by the proposed alteration in our system of government.

The friends and adversaries of the plan of the convention, if they agree in nothing else, concur at least in the value they set upon the trial by jury; or if there is any difference between them it consists in this: the former regard it as a valuable safeguard to liberty; the latter represent it as the very palladium of free government. For my own part, the more the operation of the institution has fallen under my observation, the more reason I have discovered for holding it in high estimation; and it would be altogether superfluous to examine to what extent it deserves to be esteemed useful or essential in a representative republic, or how much more merit it may be entitled to, as a defense against the oppressions of an hereditary monarch, than as a barrier to the tyranny of popular magistrates in a popular government. Discussions of this kind would be more curious than beneficial, as all are satisfied of the utility of the institution, and of its friendly aspect to liberty. But I must acknowledge that I cannot readily discern the inseparable connection between the existence of liberty, and the trial by jury in civil cases. Arbitrary impeachments, arbitrary methods of prosecuting pretended offenses, and arbitrary punishments upon arbitrary convictions, have ever appeared to me to be the great engines of judicial despotism; and these have all relation to criminal proceedings. The trial by jury in criminal cases, aided by the habeas corpus act, seems therefore to be alone concerned in the question. And both of these are provided for, in the most ample manner, in the plan of the convention.

It has been observed, that trial by jury is a safeguard against an oppressive exercise of the power of taxation. This observation deserves to be canvassed.

It is evident that it can have no influence upon the legislature, in regard to the amount of taxes to be laid, to the objects upon which they are to be imposed, or to the rule by which they are to be apportioned. If it can have any influence, therefore, it must be upon the mode of collection, and the conduct of the officers intrusted with the execution of the revenue laws.

As to the mode of collection in this State, under our own Constitution, the trial by jury is in most cases out of use. The taxes are usually levied by the more summary proceeding of distress and sale, as in cases of rent. And it is acknowledged on all hands, that this is essential to the efficacy of the revenue laws. The dilatory course of a trial at law to recover the taxes imposed on individuals, would neither suit the exigencies of the public nor promote the convenience of the citizens. It would often occasion an accumulation of costs, more burdensome than the original sum of the tax to be levied.

And as to the conduct of the officers of the revenue, the provision in favor of trial by jury in criminal cases, will afford the security aimed at. Wilful abuses of a public authority, to the oppression of the subject, and every species of official extortion, are offenses against the government, for which the persons who commit them may be indicted and punished according to the circumstances of the case.

The excellence of the trial by jury in civil cases appears to depend on circumstances foreign to the preservation of liberty. The strongest argument in its favor is, that it is a security against corruption. As there is always more time and better opportunity to tamper with a standing body of magistrates than with a jury summoned for the occasion, there is room to suppose that a corrupt influence would more easily find its way to the former than to the latter. The force of this consideration is, however, diminished by others. The sheriff, who is the summoner of ordinary juries, and the clerks of courts, who have the nomination of special juries, are themselves standing officers, and, acting individually, may be supposed more accessible to the touch of corruption than the judges, who are a collective body. It is not difficult to see, that it would be in the power of those officers to select jurors who would serve the purpose of the party as well as a corrupted bench. In the next place, it may fairly be supposed, that there would be less difficulty in gaining some of the jurors promiscuously taken from the public mass, than in gaining men who had been chosen by the government for their probity and good character. But making every deduction for these considerations, the trial by jury must still be a valuable check upon corruption. It greatly multiplies the impediments to its success. As matters now stand, it would be necessary to corrupt both court and jury; for where the jury have gone evidently wrong, the court will generally grant a new trial, and it would be in most cases of little use to practice upon the jury, unless the court could be likewise gained. Here then is a double security; and it will readily be perceived that this complicated agency tends to preserve the purity of both institutions. By increasing the obstacles to success, it discourages attempts to seduce the integrity of either. The temptations to prostitution which the judges might have to surmount, must certainly be much fewer, while the co-operation of a jury is necessary, than they might be, if they had themselves the exclusive determination of all causes.

Notwithstanding, therefore, the doubts I have expressed, as to the essentiality of trial by jury in civil cases to liberty, I admit that it is in most cases, under proper regulations, an excellent method of determining questions of property; and that on this account alone it would be entitled to a constitutional provision in its favor if it were possible to fix the limits within which it ought to be comprehended. There is, however, in all cases, great difficulty in this; and men not blinded by enthusiasm must be sensible that in a federal government, which is a composition of societies whose ideas and institutions in relation to the matter materially vary from each other, that difficulty must be not a little augmented. For my own part, at every new view I take of the subject, I become more convinced of the reality of the obstacles which, we are authoritatively informed, prevented the insertion of a provision on this head in the plan of the convention.

The great difference between the limits of the jury trial in different States is not generally understood; and as it must have considerable influence on the sentence we ought to pass upon the omission complained of in regard to this point, an explanation of it is necessary. In this State, our judicial establishments resemble, more nearly than in any other, those of Great Britain. We have courts of common law, courts of probates (analogous in certain matters to the spiritual courts in England), a court of admiralty and a court of chancery. In the courts of common law only, the trial by jury prevails, and this with some exceptions. In all the others a single judge presides, and proceeds in general either according to the course of the canon or civil law, without the aid of a jury.\footnote{It has been erroneously insinuated with regard to the court of chancery, that this court generally tries disputed facts by a jury. The truth is, that references to a jury in that court rarely happen, and are in no case necessary but where the validity of a devise of land comes into question.} In New Jersey, there is a court of chancery which proceeds like ours, but neither courts of admiralty nor of probates, in the sense in which these last are established with us. In that State the courts of common law have the cognizance of those causes which with us are determinable in the courts of admiralty and of probates, and of course the jury trial is more extensive in New Jersey than in New York. In Pennsylvania, this is perhaps still more the case, for there is no court of chancery in that State, and its common-law courts have equity jurisdiction. It has a court of admiralty, but none of probates, at least on the plan of ours. Delaware has in these respects imitated Pennsylvania. Maryland approaches more nearly to New York, as does also Virginia, except that the latter has a plurality of chancellors. North Carolina bears most affinity to Pennsylvania; South Carolina to Virginia. I believe, however, that in some of those States which have distinct courts of admiralty, the causes depending in them are triable by juries. In Georgia there are none but common-law courts, and an appeal of course lies from the verdict of one jury to another, which is called a special jury, and for which a particular mode of appointment is marked out. In Connecticut, they have no distinct courts either of chancery or of admiralty, and their courts of probates have no jurisdiction of causes. Their common-law courts have admiralty and, to a certain extent, equity jurisdiction. In cases of importance, their General Assembly is the only court of chancery. In Connecticut, therefore, the trial by jury extends in practice further than in any other State yet mentioned. Rhode Island is, I believe, in this particular, pretty much in the situation of Connecticut. Massachusetts and New Hampshire, in regard to the blending of law, equity, and admiralty jurisdictions, are in a similar predicament. In the four Eastern States, the trial by jury not only stands upon a broader foundation than in the other States, but it is attended with a peculiarity unknown, in its full extent, to any of them. There is an appeal of course from one jury to another, till there have been two verdicts out of three on one side.

From this sketch it appears that there is a material diversity, as well in the modification as in the extent of the institution of trial by jury in civil cases, in the several States; and from this fact these obvious reflections flow: first, that no general rule could have been fixed upon by the convention which would have corresponded with the circumstances of all the States; and secondly, that more or at least as much might have been hazarded by taking the system of any one State for a standard, as by omitting a provision altogether and leaving the matter, as has been done, to legislative regulation.

The propositions which have been made for supplying the omission have rather served to illustrate than to obviate the difficulty of the thing. The minority of Pennsylvania have proposed this mode of expression for the purpose--``Trial by jury shall be as heretofore"--and this I maintain would be senseless and nugatory. The United States, in their united or collective capacity, are the \textsc{object }to which all general provisions in the Constitution must necessarily be construed to refer. Now it is evident that though trial by jury, with various limitations, is known in each State individually, yet in the United States, as such, it is at this time altogether unknown, because the present federal government has no judiciary power whatever; and consequently there is no proper antecedent or previous establishment to which the term heretofore could relate. It would therefore be destitute of a precise meaning, and inoperative from its uncertainty.

As, on the one hand, the form of the provision would not fulfil the intent of its proposers, so, on the other, if I apprehend that intent rightly, it would be in itself inexpedient. I presume it to be, that causes in the federal courts should be tried by jury, if, in the State where the courts sat, that mode of trial would obtain in a similar case in the State courts; that is to say, admiralty causes should be tried in Connecticut by a jury, in New York without one. The capricious operation of so dissimilar a method of trial in the same cases, under the same government, is of itself sufficient to indispose every wellregulated judgment towards it. Whether the cause should be tried with or without a jury, would depend, in a great number of cases, on the accidental situation of the court and parties.

But this is not, in my estimation, the greatest objection. I feel a deep and deliberate conviction that there are many cases in which the trial by jury is an ineligible one. I think it so particularly in cases which concern the public peace with foreign nations--that is, in most cases where the question turns wholly on the laws of nations. Of this nature, among others, are all prize causes. Juries cannot be supposed competent to investigations that require a thorough knowledge of the laws and usages of nations; and they will sometimes be under the influence of impressions which will not suffer them to pay sufficient regard to those considerations of public policy which ought to guide their inquiries. There would of course be always danger that the rights of other nations might be infringed by their decisions, so as to afford occasions of reprisal and war. Though the proper province of juries be to determine matters of fact, yet in most cases legal consequences are complicated with fact in such a manner as to render a separation impracticable.

It will add great weight to this remark, in relation to prize causes, to mention that the method of determining them has been thought worthy of particular regulation in various treaties between different powers of Europe, and that, pursuant to such treaties, they are determinable in Great Britain, in the last resort, before the king himself, in his privy council, where the fact, as well as the law, undergoes a re-examination. This alone demonstrates the impolicy of inserting a fundamental provision in the Constitution which would make the State systems a standard for the national government in the article under consideration, and the danger of encumbering the government with any constitutional provisions the propriety of which is not indisputable.

My convictions are equally strong that great advantages result from the separation of the equity from the law jurisdiction, and that the causes which belong to the former would be improperly committed to juries. The great and primary use of a court of equity is to give relief in extraordinary cases, which are exceptions\footnote{It is true that the principles by which that relief is governed are now reduced to a regular system; but it is not the less true that they are in the main applicable to \textsc{special }circumstances, which form exceptions to general rules.} to general rules. To unite the jurisdiction of such cases with the ordinary jurisdiction, must have a tendency to unsettle the general rules, and to subject every case that arises to a special determination; while a separation of the one from the other has the contrary effect of rendering one a sentinel over the other, and of keeping each within the expedient limits. Besides this, the circumstances that constitute cases proper for courts of equity are in many instances so nice and intricate, that they are incompatible with the genius of trials by jury. They require often such long, deliberate, and critical investigation as would be impracticable to men called from their occupations, and obliged to decide before they were permitted to return to them. The simplicity and expedition which form the distinguishing characters of this mode of trial require that the matter to be decided should be reduced to some single and obvious point; while the litigations usual in chancery frequently comprehend a long train of minute and independent particulars.

It is true that the separation of the equity from the legal jurisdiction is peculiar to the English system of jurisprudence: which is the model that has been followed in several of the States. But it is equally true that the trial by jury has been unknown in every case in which they have been united. And the separation is essential to the preservation of that institution in its pristine purity. The nature of a court of equity will readily permit the extension of its jurisdiction to matters of law; but it is not a little to be suspected, that the attempt to extend the jurisdiction of the courts of law to matters of equity will not only be unproductive of the advantages which may be derived from courts of chancery, on the plan upon which they are established in this State, but will tend gradually to change the nature of the courts of law, and to undermine the trial by jury, by introducing questions too complicated for a decision in that mode.

These appeared to be conclusive reasons against incorporating the systems of all the States, in the formation of the national judiciary, according to what may be conjectured to have been the attempt of the Pennsylvania minority. Let us now examine how far the proposition of Massachusetts is calculated to remedy the supposed defect.

It is in this form: ``In civil actions between citizens of different States, every issue of fact, arising in actions at common law, may be tried by a jury if the parties, or either of them request it."

This, at best, is a proposition confined to one description of causes; and the inference is fair, either that the Massachusetts convention considered that as the only class of federal causes, in which the trial by jury would be proper; or that if desirous of a more extensive provision, they found it impracticable to devise one which would properly answer the end. If the first, the omission of a regulation respecting so partial an object can never be considered as a material imperfection in the system. If the last, it affords a strong corroboration of the extreme difficulty of the thing.

But this is not all: if we advert to the observations already made respecting the courts that subsist in the several States of the Union, and the different powers exercised by them, it will appear that there are no expressions more vague and indeterminate than those which have been employed to characterize that species of causes which it is intended shall be entitled to a trial by jury. In this State, the boundaries between actions at common law and actions of equitable jurisdiction, are ascertained in conformity to the rules which prevail in England upon that subject. In many of the other States the boundaries are less precise. In some of them every cause is to be tried in a court of common law, and upon that foundation every action may be considered as an action at common law, to be determined by a jury, if the parties, or either of them, choose it. Hence the same irregularity and confusion would be introduced by a compliance with this proposition, that I have already noticed as resulting from the regulation proposed by the Pennsylvania minority. In one State a cause would receive its determination from a jury, if the parties, or either of them, requested it; but in another State, a cause exactly similar to the other, must be decided without the intervention of a jury, because the State judicatories varied as to common-law jurisdiction.

It is obvious, therefore, that the Massachusetts proposition, upon this subject cannot operate as a general regulation, until some uniform plan, with respect to the limits of common-law and equitable jurisdictions, shall be adopted by the different States. To devise a plan of that kind is a task arduous in itself, and which it would require much time and reflection to mature. It would be extremely difficult, if not impossible, to suggest any general regulation that would be acceptable to all the States in the Union, or that would perfectly quadrate with the several State institutions.

It may be asked, Why could not a reference have been made to the constitution of this State, taking that, which is allowed by me to be a good one, as a standard for the United States? I answer that it is not very probable the other States would entertain the same opinion of our institutions as we do ourselves. It is natural to suppose that they are hitherto more attached to their own, and that each would struggle for the preference. If the plan of taking one State as a model for the whole had been thought of in the convention, it is to be presumed that the adoption of it in that body would have been rendered difficult by the predilection of each representation in favor of its own government; and it must be uncertain which of the States would have been taken as the model. It has been shown that many of them would be improper ones. And I leave it to conjecture, whether, under all circumstances, it is most likely that New York, or some other State, would have been preferred. But admit that a judicious selection could have been effected in the convention, still there would have been great danger of jealousy and disgust in the other States, at the partiality which had been shown to the institutions of one. The enemies of the plan would have been furnished with a fine pretext for raising a host of local prejudices against it, which perhaps might have hazarded, in no inconsiderable degree, its final establishment.

To avoid the embarrassments of a definition of the cases which the trial by jury ought to embrace, it is sometimes suggested by men of enthusiastic tempers, that a provision might have been inserted for establishing it in all cases whatsoever. For this I believe, no precedent is to be found in any member of the Union; and the considerations which have been stated in discussing the proposition of the minority of Pennsylvania, must satisfy every sober mind that the establishment of the trial by jury in all cases would have been an unpardonable error in the plan.

In short, the more it is considered the more arduous will appear the task of fashioning a provision in such a form as not to express too little to answer the purpose, or too much to be advisable; or which might not have opened other sources of opposition to the great and essential object of introducing a firm national government.

I cannot but persuade myself, on the other hand, that the different lights in which the subject has been placed in the course of these observations, will go far towards removing in candid minds the apprehensions they may have entertained on the point. They have tended to show that the security of liberty is materially concerned only in the trial by jury in criminal cases, which is provided for in the most ample manner in the plan of the convention; that even in far the greatest proportion of civil cases, and those in which the great body of the community is interested, that mode of trial will remain in its full force, as established in the State constitutions, untouched and unaffected by the plan of the convention; that it is in no case abolished\footnote{Vide No. 81, in which the supposition of its being abolished by the appellate jurisdiction in matters of fact being vested in the Supreme Court, is examined and refuted.} by that plan; and that there are great if not insurmountable difficulties in the way of making any precise and proper provision for it in a Constitution for the United States.

The best judges of the matter will be the least anxious for a constitutional establishment of the trial by jury in civil cases, and will be the most ready to admit that the changes which are continually happening in the affairs of society may render a different mode of determining questions of property preferable in many cases in which that mode of trial now prevails. For my part, I acknowledge myself to be convinced that even in this State it might be advantageously extended to some cases to which it does not at present apply, and might as advantageously be abridged in others. It is conceded by all reasonable men that it ought not to obtain in all cases. The examples of innovations which contract its ancient limits, as well in these States as in Great Britain, afford a strong presumption that its former extent has been found inconvenient, and give room to suppose that future experience may discover the propriety and utility of other exceptions. I suspect it to be impossible in the nature of the thing to fix the salutary point at which the operation of the institution ought to stop, and this is with me a strong argument for leaving the matter to the discretion of the legislature.

This is now clearly understood to be the case in Great Britain, and it is equally so in the State of Connecticut; and yet it may be safely affirmed that more numerous encroachments have been made upon the trial by jury in this State since the Revolution, though provided for by a positive article of our constitution, than has happened in the same time either in Connecticut or Great Britain. It may be added that these encroachments have generally originated with the men who endeavor to persuade the people they are the warmest defenders of popular liberty, but who have rarely suffered constitutional obstacles to arrest them in a favorite career. The truth is that the general \textsc{genius }of a government is all that can be substantially relied upon for permanent effects. Particular provisions, though not altogether useless, have far less virtue and efficacy than are commonly ascribed to them; and the want of them will never be, with men of sound discernment, a decisive objection to any plan which exhibits the leading characters of a good government.

It certainly sounds not a little harsh and extraordinary to affirm that there is no security for liberty in a Constitution which expressly establishes the trial by jury in criminal cases, because it does not do it in civil also; while it is a notorious fact that Connecticut, which has been always regarded as the most popular State in the Union, can boast of no constitutional provision for either.

\vspace{.5cm}
\textsc{Publius}

\vspace{1.5cm}


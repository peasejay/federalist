\chapter[No. 61: The Same Subject Continued (Concerning the Power of Congress to Regulate the Election of Members)]{No. 61\\ {\small The Same Subject Continued (Concerning the Power of Congress to Regulate the Election of Members)}}

\textit{Alexander Hamilton}

\textit{Original publication date: February 26, 1788}
\vspace{1cm}

To the People of the State of New York:
\vspace{.4cm}

\textsc{The} more candid opposers of the provision respecting elections, contained in the plan of the convention, when pressed in argument, will sometimes concede the propriety of that provision; with this qualification, however, that it ought to have been accompanied with a declaration, that all elections should be had in the counties where the electors resided. 
This, say they, was a necessary precaution against an abuse of the power. 
A declaration of this nature would certainly have been harmless; so far as it would have had the effect of quieting apprehensions, it might not have been undesirable. 
But it would, in fact, have afforded little or no additional security against the danger apprehended; and the want of it will never be considered, by an impartial and judicious examiner, as a serious, still less as an insuperable, objection to the plan. 
The different views taken of the subject in the two preceding papers must be sufficient to satisfy all dispassionate and discerning men, that if the public liberty should ever be the victim of the ambition of the national rulers, the power under examination, at least, will be guiltless of the sacrifice.

If those who are inclined to consult their jealousy only, would exercise it in a careful inspection of the several State constitutions, they would find little less room for disquietude and alarm, from the latitude which most of them allow in respect to elections, than from the latitude which is proposed to be allowed to the national government in the same respect. 
A review of their situation, in this particular, would tend greatly to remove any ill impressions which may remain in regard to this matter. 
But as that view would lead into long and tedious details, I shall content myself with the single example of the State in which I write. 
The constitution of New York makes no other provision for \textsc{locality} of elections, than that the members of the Assembly shall be elected in the \textsc{counties}; those of the Senate, in the great districts into which the State is or may be divided: these at present are four in number, and comprehend each from two to six counties. 
It may readily be perceived that it would not be more difficult to the legislature of New York to defeat the suffrages of the citizens of New York, by confining elections to particular places, than for the legislature of the United States to defeat the suffrages of the citizens of the Union, by the like expedient. 
Suppose, for instance, the city of Albany was to be appointed the sole place of election for the county and district of which it is a part, would not the inhabitants of that city speedily become the only electors of the members both of the Senate and Assembly for that county and district? 
Can we imagine that the electors who reside in the remote subdivisions of the counties of Albany, Saratoga, Cambridge, etc., or in any part of the county of Montgomery, would take the trouble to come to the city of Albany, to give their votes for members of the Assembly or Senate, sooner than they would repair to the city of New York, to participate in the choice of the members of the federal House of Representatives? 
The alarming indifference discoverable in the exercise of so invaluable a privilege under the existing laws, which afford every facility to it, furnishes a ready answer to this question. 
And, abstracted from any experience on the subject, we can be at no loss to determine, that when the place of election is at an \textsc{inconvenient distance} from the elector, the effect upon his conduct will be the same whether that distance be twenty miles or twenty thousand miles. 
Hence it must appear, that objections to the particular modification of the federal power of regulating elections will, in substance, apply with equal force to the modification of the like power in the constitution of this State; and for this reason it will be impossible to acquit the one, and to condemn the other. 
A similar comparison would lead to the same conclusion in respect to the constitutions of most of the other States.

If it should be said that defects in the State constitutions furnish no apology for those which are to be found in the plan proposed, I answer, that as the former have never been thought chargeable with inattention to the security of liberty, where the imputations thrown on the latter can be shown to be applicable to them also, the presumption is that they are rather the cavilling refinements of a predetermined opposition, than the well-founded inferences of a candid research after truth. 
To those who are disposed to consider, as innocent omissions in the State constitutions, what they regard as unpardonable blemishes in the plan of the convention, nothing can be said; or at most, they can only be asked to assign some substantial reason why the representatives of the people in a single State should be more impregnable to the lust of power, or other sinister motives, than the representatives of the people of the United States? 
If they cannot do this, they ought at least to prove to us that it is easier to subvert the liberties of three millions of people, with the advantage of local governments to head their opposition, than of two hundred thousand people who are destitute of that advantage. 
And in relation to the point immediately under consideration, they ought to convince us that it is less probable that a predominant faction in a single State should, in order to maintain its superiority, incline to a preference of a particular class of electors, than that a similar spirit should take possession of the representatives of thirteen States, spread over a vast region, and in several respects distinguishable from each other by a diversity of local circumstances, prejudices, and interests.

Hitherto my observations have only aimed at a vindication of the provision in question, on the ground of theoretic propriety, on that of the danger of placing the power elsewhere, and on that of the safety of placing it in the manner proposed. 
But there remains to be mentioned a positive advantage which will result from this disposition, and which could not as well have been obtained from any other: I allude to the circumstance of uniformity in the time of elections for the federal House of Representatives. 
It is more than possible that this uniformity may be found by experience to be of great importance to the public welfare, both as a security against the perpetuation of the same spirit in the body, and as a cure for the diseases of faction. 
If each State may choose its own time of election, it is possible there may be at least as many different periods as there are months in the year. 
The times of election in the several States, as they are now established for local purposes, vary between extremes as wide as March and November. 
The consequence of this diversity would be that there could never happen a total dissolution or renovation of the body at one time. 
If an improper spirit of any kind should happen to prevail in it, that spirit would be apt to infuse itself into the new members, as they come forward in succession. 
The mass would be likely to remain nearly the same, assimilating constantly to itself its gradual accretions. 
There is a contagion in example which few men have sufficient force of mind to resist. 
I am inclined to think that treble the duration in office, with the condition of a total dissolution of the body at the same time, might be less formidable to liberty than one third of that duration subject to gradual and successive alterations.

Uniformity in the time of elections seems not less requisite for executing the idea of a regular rotation in the Senate, and for conveniently assembling the legislature at a stated period in each year.

It may be asked, Why, then, could not a time have been fixed in the Constitution? 
As the most zealous adversaries of the plan of the convention in this State are, in general, not less zealous admirers of the constitution of the State, the question may be retorted, and it may be asked, Why was not a time for the like purpose fixed in the constitution of this State? 
No better answer can be given than that it was a matter which might safely be entrusted to legislative discretion; and that if a time had been appointed, it might, upon experiment, have been found less convenient than some other time. 
The same answer may be given to the question put on the other side. 
And it may be added that the supposed danger of a gradual change being merely speculative, it would have been hardly advisable upon that speculation to establish, as a fundamental point, what would deprive several States of the convenience of having the elections for their own governments and for the national government at the same epochs.

\vspace{.5cm}
\textsc{Publius}

\vspace{1.5cm}


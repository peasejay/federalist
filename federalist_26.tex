\chapter[No. 26: The Idea of Restraining the Legislative Authority in Regard to the Common Defense Considered.]{No. 26\\ {\small The Idea of Restraining the Legislative Authority in Regard to the Common Defense Considered.}}
To the People of the State of New York:
\vspace{.4cm}

\textsc{It was }a thing hardly to be expected that in a popular revolution the minds of men should stop at that happy mean which marks the salutary boundary between \textsc{power }and \textsc{privilege}, and combines the energy of government with the security of private rights. A failure in this delicate and important point is the great source of the inconveniences we experience, and if we are not cautious to avoid a repetition of the error, in our future attempts to rectify and ameliorate our system, we may travel from one chimerical project to another; we may try change after change; but we shall never be likely to make any material change for the better.

The idea of restraining the legislative authority, in the means of providing for the national defense, is one of those refinements which owe their origin to a zeal for liberty more ardent than enlightened. We have seen, however, that it has not had thus far an extensive prevalency; that even in this country, where it made its first appearance, Pennsylvania and North Carolina are the only two States by which it has been in any degree patronized; and that all the others have refused to give it the least countenance; wisely judging that confidence must be placed somewhere; that the necessity of doing it, is implied in the very act of delegating power; and that it is better to hazard the abuse of that confidence than to embarrass the government and endanger the public safety by impolitic restrictions on the legislative authority. The opponents of the proposed Constitution combat, in this respect, the general decision of America; and instead of being taught by experience the propriety of correcting any extremes into which we may have heretofore run, they appear disposed to conduct us into others still more dangerous, and more extravagant. As if the tone of government had been found too high, or too rigid, the doctrines they teach are calculated to induce us to depress or to relax it, by expedients which, upon other occasions, have been condemned or forborne. It may be affirmed without the imputation of invective, that if the principles they inculcate, on various points, could so far obtain as to become the popular creed, they would utterly unfit the people of this country for any species of government whatever. But a danger of this kind is not to be apprehended. The citizens of America have too much discernment to be argued into anarchy. And I am much mistaken, if experience has not wrought a deep and solemn conviction in the public mind, that greater energy of government is essential to the welfare and prosperity of the community.

It may not be amiss in this place concisely to remark the origin and progress of the idea, which aims at the exclusion of military establishments in time of peace. Though in speculative minds it may arise from a contemplation of the nature and tendency of such institutions, fortified by the events that have happened in other ages and countries, yet as a national sentiment, it must be traced to those habits of thinking which we derive from the nation from whom the inhabitants of these States have in general sprung.

In England, for a long time after the Norman Conquest, the authority of the monarch was almost unlimited. Inroads were gradually made upon the prerogative, in favor of liberty, first by the barons, and afterwards by the people, till the greatest part of its most formidable pretensions became extinct. But it was not till the revolution in 1688, which elevated the Prince of Orange to the throne of Great Britain, that English liberty was completely triumphant. As incident to the undefined power of making war, an acknowledged prerogative of the crown, Charles II. had, by his own authority, kept on foot in time of peace a body of 5,000 regular troops. And this number James II. increased to 30,000; who were paid out of his civil list. At the revolution, to abolish the exercise of so dangerous an authority, it became an article of the Bill of Rights then framed, that ``the raising or keeping a standing army within the kingdom in time of peace, \textsc{unless with the consent of parliament}, was against law."

In that kingdom, when the pulse of liberty was at its highest pitch, no security against the danger of standing armies was thought requisite, beyond a prohibition of their being raised or kept up by the mere authority of the executive magistrate. The patriots, who effected that memorable revolution, were too temperate, too wellinformed, to think of any restraint on the legislative discretion. They were aware that a certain number of troops for guards and garrisons were indispensable; that no precise bounds could be set to the national exigencies; that a power equal to every possible contingency must exist somewhere in the government: and that when they referred the exercise of that power to the judgment of the legislature, they had arrived at the ultimate point of precaution which was reconcilable with the safety of the community.

From the same source, the people of America may be said to have derived an hereditary impression of danger to liberty, from standing armies in time of peace. The circumstances of a revolution quickened the public sensibility on every point connected with the security of popular rights, and in some instances raise the warmth of our zeal beyond the degree which consisted with the due temperature of the body politic. The attempts of two of the States to restrict the authority of the legislature in the article of military establishments, are of the number of these instances. The principles which had taught us to be jealous of the power of an hereditary monarch were by an injudicious excess extended to the representatives of the people in their popular assemblies. Even in some of the States, where this error was not adopted, we find unnecessary declarations that standing armies ought not to be kept up, in time of peace, \textsc{without the consent of the legislature}. I call them unnecessary, because the reason which had introduced a similar provision into the English Bill of Rights is not applicable to any of the State constitutions. The power of raising armies at all, under those constitutions, can by no construction be deemed to reside anywhere else, than in the legislatures themselves; and it was superfluous, if not absurd, to declare that a matter should not be done without the consent of a body, which alone had the power of doing it. Accordingly, in some of these constitutions, and among others, in that of this State of New York, which has been justly celebrated, both in Europe and America, as one of the best of the forms of government established in this country, there is a total silence upon the subject.

It is remarkable, that even in the two States which seem to have meditated an interdiction of military establishments in time of peace, the mode of expression made use of is rather cautionary than prohibitory. It is not said, that standing armies \textsc{shall not be }kept up, but that they \textsc{ought not }to be kept up, in time of peace. This ambiguity of terms appears to have been the result of a conflict between jealousy and conviction; between the desire of excluding such establishments at all events, and the persuasion that an absolute exclusion would be unwise and unsafe.

Can it be doubted that such a provision, whenever the situation of public affairs was understood to require a departure from it, would be interpreted by the legislature into a mere admonition, and would be made to yield to the necessities or supposed necessities of the State? Let the fact already mentioned, with respect to Pennsylvania, decide. What then (it may be asked) is the use of such a provision, if it cease to operate the moment there is an inclination to disregard it?

Let us examine whether there be any comparison, in point of efficacy, between the provision alluded to and that which is contained in the new Constitution, for restraining the appropriations of money for military purposes to the period of two years. The former, by aiming at too much, is calculated to effect nothing; the latter, by steering clear of an imprudent extreme, and by being perfectly compatible with a proper provision for the exigencies of the nation, will have a salutary and powerful operation.

The legislature of the United States will be \textsc{obliged}, by this provision, once at least in every two years, to deliberate upon the propriety of keeping a military force on foot; to come to a new resolution on the point; and to declare their sense of the matter, by a formal vote in the face of their constituents. They are not \textsc{at liberty }to vest in the executive department permanent funds for the support of an army, if they were even incautious enough to be willing to repose in it so improper a confidence. As the spirit of party, in different degrees, must be expected to infect all political bodies, there will be, no doubt, persons in the national legislature willing enough to arraign the measures and criminate the views of the majority. The provision for the support of a military force will always be a favorable topic for declamation. As often as the question comes forward, the public attention will be roused and attracted to the subject, by the party in opposition; and if the majority should be really disposed to exceed the proper limits, the community will be warned of the danger, and will have an opportunity of taking measures to guard against it. Independent of parties in the national legislature itself, as often as the period of discussion arrived, the State legislatures, who will always be not only vigilant but suspicious and jealous guardians of the rights of the citizens against encroachments from the federal government, will constantly have their attention awake to the conduct of the national rulers, and will be ready enough, if any thing improper appears, to sound the alarm to the people, and not only to be the \textsc{voice}, but, if necessary, the \textsc{arm }of their discontent.

Schemes to subvert the liberties of a great community \textsc{require time }to mature them for execution. An army, so large as seriously to menace those liberties, could only be formed by progressive augmentations; which would suppose, not merely a temporary combination between the legislature and executive, but a continued conspiracy for a series of time. Is it probable that such a combination would exist at all? Is it probable that it would be persevered in, and transmitted along through all the successive variations in a representative body, which biennial elections would naturally produce in both houses? Is it presumable, that every man, the instant he took his seat in the national Senate or House of Representatives, would commence a traitor to his constituents and to his country? Can it be supposed that there would not be found one man, discerning enough to detect so atrocious a conspiracy, or bold or honest enough to apprise his constituents of their danger? If such presumptions can fairly be made, there ought at once to be an end of all delegated authority. The people should resolve to recall all the powers they have heretofore parted with out of their own hands, and to divide themselves into as many States as there are counties, in order that they may be able to manage their own concerns in person.

If such suppositions could even be reasonably made, still the concealment of the design, for any duration, would be impracticable. It would be announced, by the very circumstance of augmenting the army to so great an extent in time of profound peace. What colorable reason could be assigned, in a country so situated, for such vast augmentations of the military force? It is impossible that the people could be long deceived; and the destruction of the project, and of the projectors, would quickly follow the discovery.

It has been said that the provision which limits the appropriation of money for the support of an army to the period of two years would be unavailing, because the Executive, when once possessed of a force large enough to awe the people into submission, would find resources in that very force sufficient to enable him to dispense with supplies from the acts of the legislature. But the question again recurs, upon what pretense could he be put in possession of a force of that magnitude in time of peace? If we suppose it to have been created in consequence of some domestic insurrection or foreign war, then it becomes a case not within the principles of the objection; for this is levelled against the power of keeping up troops in time of peace. Few persons will be so visionary as seriously to contend that military forces ought not to be raised to quell a rebellion or resist an invasion; and if the defense of the community under such circumstances should make it necessary to have an army so numerous as to hazard its liberty, this is one of those calamities for which there is neither preventative nor cure. It cannot be provided against by any possible form of government; it might even result from a simple league offensive and defensive, if it should ever be necessary for the confederates or allies to form an army for common defense.

But it is an evil infinitely less likely to attend us in a united than in a disunited state; nay, it may be safely asserted that it is an evil altogether unlikely to attend us in the latter situation. It is not easy to conceive a possibility that dangers so formidable can assail the whole Union, as to demand a force considerable enough to place our liberties in the least jeopardy, especially if we take into our view the aid to be derived from the militia, which ought always to be counted upon as a valuable and powerful auxiliary. But in a state of disunion (as has been fully shown in another place), the contrary of this supposition would become not only probable, but almost unavoidable.

\vspace{.5cm}
\textsc{Publius}

\vspace{1.5cm}


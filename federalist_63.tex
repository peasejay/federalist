\chapter[No. 63: The Senate Continued]{No. 63\\ {\small The Senate Continued}}
To the People of the State of New York:
\vspace{.4cm}

\textsc{A fifth }desideratum, illustrating the utility of a senate, is the want of a due sense of national character. Without a select and stable member of the government, the esteem of foreign powers will not only be forfeited by an unenlightened and variable policy, proceeding from the causes already mentioned, but the national councils will not possess that sensibility to the opinion of the world, which is perhaps not less necessary in order to merit, than it is to obtain, its respect and confidence.

An attention to the judgment of other nations is important to every government for two reasons: the one is, that, independently of the merits of any particular plan or measure, it is desirable, on various accounts, that it should appear to other nations as the offspring of a wise and honorable policy; the second is, that in doubtful cases, particularly where the national councils may be warped by some strong passion or momentary interest, the presumed or known opinion of the impartial world may be the best guide that can be followed. What has not America lost by her want of character with foreign nations; and how many errors and follies would she not have avoided, if the justice and propriety of her measures had, in every instance, been previously tried by the light in which they would probably appear to the unbiased part of mankind?

Yet however requisite a sense of national character may be, it is evident that it can never be sufficiently possessed by a numerous and changeable body. It can only be found in a number so small that a sensible degree of the praise and blame of public measures may be the portion of each individual; or in an assembly so durably invested with public trust, that the pride and consequence of its members may be sensibly incorporated with the reputation and prosperity of the community. The half-yearly representatives of Rhode Island would probably have been little affected in their deliberations on the iniquitous measures of that State, by arguments drawn from the light in which such measures would be viewed by foreign nations, or even by the sister States; whilst it can scarcely be doubted that if the concurrence of a select and stable body had been necessary, a regard to national character alone would have prevented the calamities under which that misguided people is now laboring.

I add, as a \textsc{sixth }defect the want, in some important cases, of a due responsibility in the government to the people, arising from that frequency of elections which in other cases produces this responsibility. This remark will, perhaps, appear not only new, but paradoxical. It must nevertheless be acknowledged, when explained, to be as undeniable as it is important.

Responsibility, in order to be reasonable, must be limited to objects within the power of the responsible party, and in order to be effectual, must relate to operations of that power, of which a ready and proper judgment can be formed by the constituents. The objects of government may be divided into two general classes: the one depending on measures which have singly an immediate and sensible operation; the other depending on a succession of well-chosen and well-connected measures, which have a gradual and perhaps unobserved operation. The importance of the latter description to the collective and permanent welfare of every country, needs no explanation. And yet it is evident that an assembly elected for so short a term as to be unable to provide more than one or two links in a chain of measures, on which the general welfare may essentially depend, ought not to be answerable for the final result, any more than a steward or tenant, engaged for one year, could be justly made to answer for places or improvements which could not be accomplished in less than half a dozen years. Nor is it possible for the people to estimate the \textsc{share }of influence which their annual assemblies may respectively have on events resulting from the mixed transactions of several years. It is sufficiently difficult to preserve a personal responsibility in the members of a \textsc{numerous }body, for such acts of the body as have an immediate, detached, and palpable operation on its constituents.

The proper remedy for this defect must be an additional body in the legislative department, which, having sufficient permanency to provide for such objects as require a continued attention, and a train of measures, may be justly and effectually answerable for the attainment of those objects.

Thus far I have considered the circumstances which point out the necessity of a well-constructed Senate only as they relate to the representatives of the people. To a people as little blinded by prejudice or corrupted by flattery as those whom I address, I shall not scruple to add, that such an institution may be sometimes necessary as a defense to the people against their own temporary errors and delusions. As the cool and deliberate sense of the community ought, in all governments, and actually will, in all free governments, ultimately prevail over the views of its rulers; so there are particular moments in public affairs when the people, stimulated by some irregular passion, or some illicit advantage, or misled by the artful misrepresentations of interested men, may call for measures which they themselves will afterwards be the most ready to lament and condemn. In these critical moments, how salutary will be the interference of some temperate and respectable body of citizens, in order to check the misguided career, and to suspend the blow meditated by the people against themselves, until reason, justice, and truth can regain their authority over the public mind? What bitter anguish would not the people of Athens have often escaped if their government had contained so provident a safeguard against the tyranny of their own passions? Popular liberty might then have escaped the indelible reproach of decreeing to the same citizens the hemlock on one day and statues on the next.

It may be suggested, that a people spread over an extensive region cannot, like the crowded inhabitants of a small district, be subject to the infection of violent passions, or to the danger of combining in pursuit of unjust measures. I am far from denying that this is a distinction of peculiar importance. I have, on the contrary, endeavored in a former paper to show, that it is one of the principal recommendations of a confederated republic. At the same time, this advantage ought not to be considered as superseding the use of auxiliary precautions. It may even be remarked, that the same extended situation, which will exempt the people of America from some of the dangers incident to lesser republics, will expose them to the inconveniency of remaining for a longer time under the influence of those misrepresentations which the combined industry of interested men may succeed in distributing among them.

It adds no small weight to all these considerations, to recollect that history informs us of no long-lived republic which had not a senate. Sparta, Rome, and Carthage are, in fact, the only states to whom that character can be applied. In each of the two first there was a senate for life. The constitution of the senate in the last is less known. Circumstantial evidence makes it probable that it was not different in this particular from the two others. It is at least certain, that it had some quality or other which rendered it an anchor against popular fluctuations; and that a smaller council, drawn out of the senate, was appointed not only for life, but filled up vacancies itself. These examples, though as unfit for the imitation, as they are repugnant to the genius, of America, are, notwithstanding, when compared with the fugitive and turbulent existence of other ancient republics, very instructive proofs of the necessity of some institution that will blend stability with liberty. I am not unaware of the circumstances which distinguish the American from other popular governments, as well ancient as modern; and which render extreme circumspection necessary, in reasoning from the one case to the other. But after allowing due weight to this consideration, it may still be maintained, that there are many points of similitude which render these examples not unworthy of our attention. Many of the defects, as we have seen, which can only be supplied by a senatorial institution, are common to a numerous assembly frequently elected by the people, and to the people themselves. There are others peculiar to the former, which require the control of such an institution. The people can never wilfully betray their own interests; but they may possibly be betrayed by the representatives of the people; and the danger will be evidently greater where the whole legislative trust is lodged in the hands of one body of men, than where the concurrence of separate and dissimilar bodies is required in every public act.

The difference most relied on, between the American and other republics, consists in the principle of representation; which is the pivot on which the former move, and which is supposed to have been unknown to the latter, or at least to the ancient part of them. The use which has been made of this difference, in reasonings contained in former papers, will have shown that I am disposed neither to deny its existence nor to undervalue its importance. I feel the less restraint, therefore, in observing, that the position concerning the ignorance of the ancient governments on the subject of representation, is by no means precisely true in the latitude commonly given to it. Without entering into a disquisition which here would be misplaced, I will refer to a few known facts, in support of what I advance.

In the most pure democracies of Greece, many of the executive functions were performed, not by the people themselves, but by officers elected by the people, and \textsc{representing }the people in their \textsc{executive }capacity.

Prior to the reform of Solon, Athens was governed by nine Archons, annually \textsc{elected by the people at large}. The degree of power delegated to them seems to be left in great obscurity. Subsequent to that period, we find an assembly, first of four, and afterwards of six hundred members, annually \textsc{elected by the people}; and \textsc{partially }representing them in their \textsc{legislative }capacity, since they were not only associated with the people in the function of making laws, but had the exclusive right of originating legislative propositions to the people. The senate of Carthage, also, whatever might be its power, or the duration of its appointment, appears to have been \textsc{elective }by the suffrages of the people. Similar instances might be traced in most, if not all the popular governments of antiquity.

Lastly, in Sparta we meet with the Ephori, and in Rome with the Tribunes; two bodies, small indeed in numbers, but annually \textsc{elected by the whole body of the people}, and considered as the \textsc{representatives }of the people, almost in their \textsc{plenipotentiary }capacity. The Cosmi of Crete were also annually \textsc{elected by the people}, and have been considered by some authors as an institution analogous to those of Sparta and Rome, with this difference only, that in the election of that representative body the right of suffrage was communicated to a part only of the people.

From these facts, to which many others might be added, it is clear that the principle of representation was neither unknown to the ancients nor wholly overlooked in their political constitutions. The true distinction between these and the American governments, lies \textsc{in the total exclusion of the people}, \textsc{in their collective capacity}, from any share in the \textsc{latter}, and not in the \textsc{total exclusion of the representatives of the people }from the administration of the \textsc{former}. The distinction, however, thus qualified, must be admitted to leave a most advantageous superiority in favor of the United States. But to insure to this advantage its full effect, we must be careful not to separate it from the other advantage, of an extensive territory. For it cannot be believed, that any form of representative government could have succeeded within the narrow limits occupied by the democracies of Greece.

In answer to all these arguments, suggested by reason, illustrated by examples, and enforced by our own experience, the jealous adversary of the Constitution will probably content himself with repeating, that a senate appointed not immediately by the people, and for the term of six years, must gradually acquire a dangerous pre-eminence in the government, and finally transform it into a tyrannical aristocracy.

To this general answer, the general reply ought to be sufficient, that liberty may be endangered by the abuses of liberty as well as by the abuses of power; that there are numerous instances of the former as well as of the latter; and that the former, rather than the latter, are apparently most to be apprehended by the United States. But a more particular reply may be given.

Before such a revolution can be effected, the Senate, it is to be observed, must in the first place corrupt itself; must next corrupt the State legislatures; must then corrupt the House of Representatives; and must finally corrupt the people at large. It is evident that the Senate must be first corrupted before it can attempt an establishment of tyranny. Without corrupting the State legislatures, it cannot prosecute the attempt, because the periodical change of members would otherwise regenerate the whole body. Without exerting the means of corruption with equal success on the House of Representatives, the opposition of that coequal branch of the government would inevitably defeat the attempt; and without corrupting the people themselves, a succession of new representatives would speedily restore all things to their pristine order. Is there any man who can seriously persuade himself that the proposed Senate can, by any possible means within the compass of human address, arrive at the object of a lawless ambition, through all these obstructions?

If reason condemns the suspicion, the same sentence is pronounced by experience. The constitution of Maryland furnishes the most apposite example. The Senate of that State is elected, as the federal Senate will be, indirectly by the people, and for a term less by one year only than the federal Senate. It is distinguished, also, by the remarkable prerogative of filling up its own vacancies within the term of its appointment, and, at the same time, is not under the control of any such rotation as is provided for the federal Senate. There are some other lesser distinctions, which would expose the former to colorable objections, that do not lie against the latter. If the federal Senate, therefore, really contained the danger which has been so loudly proclaimed, some symptoms at least of a like danger ought by this time to have been betrayed by the Senate of Maryland, but no such symptoms have appeared. On the contrary, the jealousies at first entertained by men of the same description with those who view with terror the correspondent part of the federal Constitution, have been gradually extinguished by the progress of the experiment; and the Maryland constitution is daily deriving, from the salutary operation of this part of it, a reputation in which it will probably not be rivalled by that of any State in the Union.

But if anything could silence the jealousies on this subject, it ought to be the British example. The Senate there instead of being elected for a term of six years, and of being unconfined to particular families or fortunes, is an hereditary assembly of opulent nobles. The House of Representatives, instead of being elected for two years, and by the whole body of the people, is elected for seven years, and, in very great proportion, by a very small proportion of the people. Here, unquestionably, ought to be seen in full display the aristocratic usurpations and tyranny which are at some future period to be exemplified in the United States. Unfortunately, however, for the anti-federal argument, the British history informs us that this hereditary assembly has not been able to defend itself against the continual encroachments of the House of Representatives; and that it no sooner lost the support of the monarch, than it was actually crushed by the weight of the popular branch.

As far as antiquity can instruct us on this subject, its examples support the reasoning which we have employed. In Sparta, the Ephori, the annual representatives of the people, were found an overmatch for the senate for life, continually gained on its authority and finally drew all power into their own hands. The Tribunes of Rome, who were the representatives of the people, prevailed, it is well known, in almost every contest with the senate for life, and in the end gained the most complete triumph over it. The fact is the more remarkable, as unanimity was required in every act of the Tribunes, even after their number was augmented to ten. It proves the irresistible force possessed by that branch of a free government, which has the people on its side. To these examples might be added that of Carthage, whose senate, according to the testimony of Polybius, instead of drawing all power into its vortex, had, at the commencement of the second Punic War, lost almost the whole of its original portion.

Besides the conclusive evidence resulting from this assemblage of facts, that the federal Senate will never be able to transform itself, by gradual usurpations, into an independent and aristocratic body, we are warranted in believing, that if such a revolution should ever happen from causes which the foresight of man cannot guard against, the House of Representatives, with the people on their side, will at all times be able to bring back the Constitution to its primitive form and principles. Against the force of the immediate representatives of the people, nothing will be able to maintain even the constitutional authority of the Senate, but such a display of enlightened policy, and attachment to the public good, as will divide with that branch of the legislature the affections and support of the entire body of the people themselves.

\vspace{.5cm}
\textsc{Publius}

\vspace{1.5cm}


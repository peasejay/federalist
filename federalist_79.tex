\chapter[No. 79: The Judiciary Continued]{No. 79\\ {\small The Judiciary Continued}}
To the People of the State of New York:
\vspace{.4cm}

\textsc{Next} to permanency in office, nothing can contribute more to the independence of the judges than a fixed provision for their support. The remark made in relation to the President is equally applicable here. In the general course of human nature, a power over a man's subsistence amounts to a power over his will. And we can never hope to see realized in practice, the complete separation of the judicial from the legislative power, in any system which leaves the former dependent for pecuniary resources on the occasional grants of the latter. The enlightened friends to good government in every State, have seen cause to lament the want of precise and explicit precautions in the State constitutions on this head. Some of these indeed have declared that permanent\footnote{Vide Constitution of Massachusetts, Chapter 2, Section 1, Article 13.} salaries should be established for the judges; but the experiment has in some instances shown that such expressions are not sufficiently definite to preclude legislative evasions. Something still more positive and unequivocal has been evinced to be requisite. The plan of the convention accordingly has provided that the judges of the United States ``shall at stated times receive for their services a compensation which shall not be diminished during their continuance in office."

This, all circumstances considered, is the most eligible provision that could have been devised. It will readily be understood that the fluctuations in the value of money and in the state of society rendered a fixed rate of compensation in the Constitution inadmissible. What might be extravagant to-day, might in half a century become penurious and inadequate. It was therefore necessary to leave it to the discretion of the legislature to vary its provisions in conformity to the variations in circumstances, yet under such restrictions as to put it out of the power of that body to change the condition of the individual for the worse. A man may then be sure of the ground upon which he stands, and can never be deterred from his duty by the apprehension of being placed in a less eligible situation. The clause which has been quoted combines both advantages. The salaries of judicial officers may from time to time be altered, as occasion shall require, yet so as never to lessen the allowance with which any particular judge comes into office, in respect to him. It will be observed that a difference has been made by the convention between the compensation of the President and of the judges, That of the former can neither be increased nor diminished; that of the latter can only not be diminished. This probably arose from the difference in the duration of the respective offices. As the President is to be elected for no more than four years, it can rarely happen that an adequate salary, fixed at the commencement of that period, will not continue to be such to its end. But with regard to the judges, who, if they behave properly, will be secured in their places for life, it may well happen, especially in the early stages of the government, that a stipend, which would be very sufficient at their first appointment, would become too small in the progress of their service.

This provision for the support of the judges bears every mark of prudence and efficacy; and it may be safely affirmed that, together with the permanent tenure of their offices, it affords a better prospect of their independence than is discoverable in the constitutions of any of the States in regard to their own judges.

The precautions for their responsibility are comprised in the article respecting impeachments. They are liable to be impeached for malconduct by the House of Representatives, and tried by the Senate; and, if convicted, may be dismissed from office, and disqualified for holding any other. This is the only provision on the point which is consistent with the necessary independence of the judicial character, and is the only one which we find in our own Constitution in respect to our own judges.

The want of a provision for removing the judges on account of inability has been a subject of complaint. But all considerate men will be sensible that such a provision would either not be practiced upon or would be more liable to abuse than calculated to answer any good purpose. The mensuration of the faculties of the mind has, I believe, no place in the catalogue of known arts. An attempt to fix the boundary between the regions of ability and inability, would much oftener give scope to personal and party attachments and enmities than advance the interests of justice or the public good. The result, except in the case of insanity, must for the most part be arbitrary; and insanity, without any formal or express provision, may be safely pronounced to be a virtual disqualification.

The constitution of New York, to avoid investigations that must forever be vague and dangerous, has taken a particular age as the criterion of inability. No man can be a judge beyond sixty. I believe there are few at present who do not disapprove of this provision. There is no station, in relation to which it is less proper than to that of a judge. The deliberating and comparing faculties generally preserve their strength much beyond that period in men who survive it; and when, in addition to this circumstance, we consider how few there are who outlive the season of intellectual vigor, and how improbable it is that any considerable portion of the bench, whether more or less numerous, should be in such a situation at the same time, we shall be ready to conclude that limitations of this sort have little to recommend them. In a republic, where fortunes are not affluent, and pensions not expedient, the dismission of men from stations in which they have served their country long and usefully, on which they depend for subsistence, and from which it will be too late to resort to any other occupation for a livelihood, ought to have some better apology to humanity than is to be found in the imaginary danger of a superannuated bench.

\vspace{.5cm}
\textsc{Publius}

\vspace{1.5cm}


\chapter[No. 57: The Alleged Tendency of the New Plan to Elevate the Few at the Expense of the Many Considered in Connection with Representation.]{No. 57\\ {\small The Alleged Tendency of the New Plan to Elevate the Few at the Expense of the Many Considered in Connection with Representation.}}
To the People of the State of New York:
\vspace{.25cm}

\textsc{The third }charge against the House of Representatives is, that it will be taken from that class of citizens which will have least sympathy with the mass of the people, and be most likely to aim at an ambitious sacrifice of the many to the aggrandizement of the few.

Of all the objections which have been framed against the federal Constitution, this is perhaps the most extraordinary. Whilst the objection itself is levelled against a pretended oligarchy, the principle of it strikes at the very root of republican government.

The aim of every political constitution is, or ought to be, first to obtain for rulers men who possess most wisdom to discern, and most virtue to pursue, the common good of the society; and in the next place, to take the most effectual precautions for keeping them virtuous whilst they continue to hold their public trust. The elective mode of obtaining rulers is the characteristic policy of republican government. The means relied on in this form of government for preventing their degeneracy are numerous and various. The most effectual one, is such a limitation of the term of appointments as will maintain a proper responsibility to the people.

Let me now ask what circumstance there is in the constitution of the House of Representatives that violates the principles of republican government, or favors the elevation of the few on the ruins of the many? Let me ask whether every circumstance is not, on the contrary, strictly conformable to these principles, and scrupulously impartial to the rights and pretensions of every class and description of citizens?

Who are to be the electors of the federal representatives? Not the rich, more than the poor; not the learned, more than the ignorant; not the haughty heirs of distinguished names, more than the humble sons of obscurity and unpropitious fortune. The electors are to be the great body of the people of the United States. They are to be the same who exercise the right in every State of electing the corresponding branch of the legislature of the State.

Who are to be the objects of popular choice? Every citizen whose merit may recommend him to the esteem and confidence of his country. No qualification of wealth, of birth, of religious faith, or of civil profession is permitted to fetter the judgement or disappoint the inclination of the people.

If we consider the situation of the men on whom the free suffrages of their fellow-citizens may confer the representative trust, we shall find it involving every security which can be devised or desired for their fidelity to their constituents.

In the first place, as they will have been distinguished by the preference of their fellow-citizens, we are to presume that in general they will be somewhat distinguished also by those qualities which entitle them to it, and which promise a sincere and scrupulous regard to the nature of their engagements.

In the second place, they will enter into the public service under circumstances which cannot fail to produce a temporary affection at least to their constituents. There is in every breast a sensibility to marks of honor, of favor, of esteem, and of confidence, which, apart from all considerations of interest, is some pledge for grateful and benevolent returns. Ingratitude is a common topic of declamation against human nature; and it must be confessed that instances of it are but too frequent and flagrant, both in public and in private life. But the universal and extreme indignation which it inspires is itself a proof of the energy and prevalence of the contrary sentiment.

In the third place, those ties which bind the representative to his constituents are strengthened by motives of a more selfish nature. His pride and vanity attach him to a form of government which favors his pretensions and gives him a share in its honors and distinctions. Whatever hopes or projects might be entertained by a few aspiring characters, it must generally happen that a great proportion of the men deriving their advancement from their influence with the people, would have more to hope from a preservation of the favor, than from innovations in the government subversive of the authority of the people.

All these securities, however, would be found very insufficient without the restraint of frequent elections. Hence, in the fourth place, the House of Representatives is so constituted as to support in the members an habitual recollection of their dependence on the people. Before the sentiments impressed on their minds by the mode of their elevation can be effaced by the exercise of power, they will be compelled to anticipate the moment when their power is to cease, when their exercise of it is to be reviewed, and when they must descend to the level from which they were raised; there forever to remain unless a faithful discharge of their trust shall have established their title to a renewal of it.

I will add, as a fifth circumstance in the situation of the House of Representatives, restraining them from oppressive measures, that they can make no law which will not have its full operation on themselves and their friends, as well as on the great mass of the society. This has always been deemed one of the strongest bonds by which human policy can connect the rulers and the people together. It creates between them that communion of interests and sympathy of sentiments, of which few governments have furnished examples; but without which every government degenerates into tyranny. If it be asked, what is to restrain the House of Representatives from making legal discriminations in favor of themselves and a particular class of the society? I answer: the genius of the whole system; the nature of just and constitutional laws; and above all, the vigilant and manly spirit which actuates the people of America--a spirit which nourishes freedom, and in return is nourished by it.

If this spirit shall ever be so far debased as to tolerate a law not obligatory on the legislature, as well as on the people, the people will be prepared to tolerate any thing but liberty.

Such will be the relation between the House of Representatives and their constituents. Duty, gratitude, interest, ambition itself, are the chords by which they will be bound to fidelity and sympathy with the great mass of the people. It is possible that these may all be insufficient to control the caprice and wickedness of man. But are they not all that government will admit, and that human prudence can devise? Are they not the genuine and the characteristic means by which republican government provides for the liberty and happiness of the people? Are they not the identical means on which every State government in the Union relies for the attainment of these important ends? What then are we to understand by the objection which this paper has combated? What are we to say to the men who profess the most flaming zeal for republican government, yet boldly impeach the fundamental principle of it; who pretend to be champions for the right and the capacity of the people to choose their own rulers, yet maintain that they will prefer those only who will immediately and infallibly betray the trust committed to them?

Were the objection to be read by one who had not seen the mode prescribed by the Constitution for the choice of representatives, he could suppose nothing less than that some unreasonable qualification of property was annexed to the right of suffrage; or that the right of eligibility was limited to persons of particular families or fortunes; or at least that the mode prescribed by the State constitutions was in some respect or other, very grossly departed from. We have seen how far such a supposition would err, as to the two first points. Nor would it, in fact, be less erroneous as to the last. The only difference discoverable between the two cases is, that each representative of the United States will be elected by five or six thousand citizens; whilst in the individual States, the election of a representative is left to about as many hundreds. Will it be pretended that this difference is sufficient to justify an attachment to the State governments, and an abhorrence to the federal government? If this be the point on which the objection turns, it deserves to be examined.

Is it supported by \textsc{reason}? This cannot be said, without maintaining that five or six thousand citizens are less capable of choosing a fit representative, or more liable to be corrupted by an unfit one, than five or six hundred. Reason, on the contrary, assures us, that as in so great a number a fit representative would be most likely to be found, so the choice would be less likely to be diverted from him by the intrigues of the ambitious or the ambitious or the bribes of the rich.

Is the \textsc{consequence }from this doctrine admissible? If we say that five or six hundred citizens are as many as can jointly exercise their right of suffrage, must we not deprive the people of the immediate choice of their public servants, in every instance where the administration of the government does not require as many of them as will amount to one for that number of citizens?

Is the doctrine warranted by \textsc{facts}? It was shown in the last paper, that the real representation in the British House of Commons very little exceeds the proportion of one for every thirty thousand inhabitants. Besides a variety of powerful causes not existing here, and which favor in that country the pretensions of rank and wealth, no person is eligible as a representative of a county, unless he possess real estate of the clear value of six hundred pounds sterling per year; nor of a city or borough, unless he possess a like estate of half that annual value. To this qualification on the part of the county representatives is added another on the part of the county electors, which restrains the right of suffrage to persons having a freehold estate of the annual value of more than twenty pounds sterling, according to the present rate of money. Notwithstanding these unfavorable circumstances, and notwithstanding some very unequal laws in the British code, it cannot be said that the representatives of the nation have elevated the few on the ruins of the many.

But we need not resort to foreign experience on this subject. Our own is explicit and decisive. The districts in New Hampshire in which the senators are chosen immediately by the people, are nearly as large as will be necessary for her representatives in the Congress. Those of Massachusetts are larger than will be necessary for that purpose; and those of New York still more so. In the last State the members of Assembly for the cities and counties of New York and Albany are elected by very nearly as many voters as will be entitled to a representative in the Congress, calculating on the number of sixty-five representatives only. It makes no difference that in these senatorial districts and counties a number of representatives are voted for by each elector at the same time. If the same electors at the same time are capable of choosing four or five representatives, they cannot be incapable of choosing one. Pennsylvania is an additional example. Some of her counties, which elect her State representatives, are almost as large as her districts will be by which her federal representatives will be elected. The city of Philadelphia is supposed to contain between fifty and sixty thousand souls. It will therefore form nearly two districts for the choice of federal representatives. It forms, however, but one county, in which every elector votes for each of its representatives in the State legislature. And what may appear to be still more directly to our purpose, the whole city actually elects a \textsc{single member }for the executive council. This is the case in all the other counties of the State.

Are not these facts the most satisfactory proofs of the fallacy which has been employed against the branch of the federal government under consideration? Has it appeared on trial that the senators of New Hampshire, Massachusetts, and New York, or the executive council of Pennsylvania, or the members of the Assembly in the two last States, have betrayed any peculiar disposition to sacrifice the many to the few, or are in any respect less worthy of their places than the representatives and magistrates appointed in other States by very small divisions of the people?

But there are cases of a stronger complexion than any which I have yet quoted. One branch of the legislature of Connecticut is so constituted that each member of it is elected by the whole State. So is the governor of that State, of Massachusetts, and of this State, and the president of New Hampshire. I leave every man to decide whether the result of any one of these experiments can be said to countenance a suspicion, that a diffusive mode of choosing representatives of the people tends to elevate traitors and to undermine the public liberty.

\vspace{.5cm}
\textsc{Publius}

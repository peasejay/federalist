\chapter[No. 2: Concerning Dangers from Foreign Force and Influence]{No. 2\\ {\small Concerning Dangers from Foreign Force and Influence}}
To the People of the State of New York:
\vspace{.4cm}

\textsc{When }the people of America reflect that they are now called upon to decide a question, which, in its consequences, must prove one of the most important that ever engaged their attention, the propriety of their taking a very comprehensive, as well as a very serious, view of it, will be evident.

Nothing is more certain than the indispensable necessity of government, and it is equally undeniable, that whenever and however it is instituted, the people must cede to it some of their natural rights in order to vest it with requisite powers. It is well worthy of consideration therefore, whether it would conduce more to the interest of the people of America that they should, to all general purposes, be one nation, under one federal government, or that they should divide themselves into separate confederacies, and give to the head of each the same kind of powers which they are advised to place in one national government.

It has until lately been a received and uncontradicted opinion that the prosperity of the people of America depended on their continuing firmly united, and the wishes, prayers, and efforts of our best and wisest citizens have been constantly directed to that object. But politicians now appear, who insist that this opinion is erroneous, and that instead of looking for safety and happiness in union, we ought to seek it in a division of the States into distinct confederacies or sovereignties. However extraordinary this new doctrine may appear, it nevertheless has its advocates; and certain characters who were much opposed to it formerly, are at present of the number. Whatever may be the arguments or inducements which have wrought this change in the sentiments and declarations of these gentlemen, it certainly would not be wise in the people at large to adopt these new political tenets without being fully convinced that they are founded in truth and sound policy.

It has often given me pleasure to observe that independent America was not composed of detached and distant territories, but that one connected, fertile, wide-spreading country was the portion of our western sons of liberty. Providence has in a particular manner blessed it with a variety of soils and productions, and watered it with innumerable streams, for the delight and accommodation of its inhabitants. A succession of navigable waters forms a kind of chain round its borders, as if to bind it together; while the most noble rivers in the world, running at convenient distances, present them with highways for the easy communication of friendly aids, and the mutual transportation and exchange of their various commodities.

With equal pleasure I have as often taken notice that Providence has been pleased to give this one connected country to one united people--a people descended from the same ancestors, speaking the same language, professing the same religion, attached to the same principles of government, very similar in their manners and customs, and who, by their joint counsels, arms, and efforts, fighting side by side throughout a long and bloody war, have nobly established general liberty and independence.

This country and this people seem to have been made for each other, and it appears as if it was the design of Providence, that an inheritance so proper and convenient for a band of brethren, united to each other by the strongest ties, should never be split into a number of unsocial, jealous, and alien sovereignties.

Similar sentiments have hitherto prevailed among all orders and denominations of men among us. To all general purposes we have uniformly been one people each individual citizen everywhere enjoying the same national rights, privileges, and protection. As a nation we have made peace and war; as a nation we have vanquished our common enemies; as a nation we have formed alliances, and made treaties, and entered into various compacts and conventions with foreign states.

A strong sense of the value and blessings of union induced the people, at a very early period, to institute a federal government to preserve and perpetuate it. They formed it almost as soon as they had a political existence; nay, at a time when their habitations were in flames, when many of their citizens were bleeding, and when the progress of hostility and desolation left little room for those calm and mature inquiries and reflections which must ever precede the formation of a wise and well-balanced government for a free people. It is not to be wondered at, that a government instituted in times so inauspicious, should on experiment be found greatly deficient and inadequate to the purpose it was intended to answer.

This intelligent people perceived and regretted these defects. Still continuing no less attached to union than enamored of liberty, they observed the danger which immediately threatened the former and more remotely the latter; and being persuaded that ample security for both could only be found in a national government more wisely framed, they as with one voice, convened the late convention at Philadelphia, to take that important subject under consideration.

This convention composed of men who possessed the confidence of the people, and many of whom had become highly distinguished by their patriotism, virtue and wisdom, in times which tried the minds and hearts of men, undertook the arduous task. In the mild season of peace, with minds unoccupied by other subjects, they passed many months in cool, uninterrupted, and daily consultation; and finally, without having been awed by power, or influenced by any passions except love for their country, they presented and recommended to the people the plan produced by their joint and very unanimous councils.

Admit, for so is the fact, that this plan is only \textsc{recommended}, not imposed, yet let it be remembered that it is neither recommended to \textsc{blind }approbation, nor to \textsc{blind }reprobation; but to that sedate and candid consideration which the magnitude and importance of the subject demand, and which it certainly ought to receive. But this (as was remarked in the foregoing number of this paper) is more to be wished than expected, that it may be so considered and examined. Experience on a former occasion teaches us not to be too sanguine in such hopes. It is not yet forgotten that well-grounded apprehensions of imminent danger induced the people of America to form the memorable Congress of 1774. That body recommended certain measures to their constituents, and the event proved their wisdom; yet it is fresh in our memories how soon the press began to teem with pamphlets and weekly papers against those very measures. Not only many of the officers of government, who obeyed the dictates of personal interest, but others, from a mistaken estimate of consequences, or the undue influence of former attachments, or whose ambition aimed at objects which did not correspond with the public good, were indefatigable in their efforts to persuade the people to reject the advice of that patriotic Congress. Many, indeed, were deceived and deluded, but the great majority of the people reasoned and decided judiciously; and happy they are in reflecting that they did so.

They considered that the Congress was composed of many wise and experienced men. That, being convened from different parts of the country, they brought with them and communicated to each other a variety of useful information. That, in the course of the time they passed together in inquiring into and discussing the true interests of their country, they must have acquired very accurate knowledge on that head. That they were individually interested in the public liberty and prosperity, and therefore that it was not less their inclination than their duty to recommend only such measures as, after the most mature deliberation, they really thought prudent and advisable.

These and similar considerations then induced the people to rely greatly on the judgment and integrity of the Congress; and they took their advice, notwithstanding the various arts and endeavors used to deter them from it. But if the people at large had reason to confide in the men of that Congress, few of whom had been fully tried or generally known, still greater reason have they now to respect the judgment and advice of the convention, for it is well known that some of the most distinguished members of that Congress, who have been since tried and justly approved for patriotism and abilities, and who have grown old in acquiring political information, were also members of this convention, and carried into it their accumulated knowledge and experience.

It is worthy of remark that not only the first, but every succeeding Congress, as well as the late convention, have invariably joined with the people in thinking that the prosperity of America depended on its Union. To preserve and perpetuate it was the great object of the people in forming that convention, and it is also the great object of the plan which the convention has advised them to adopt. With what propriety, therefore, or for what good purposes, are attempts at this particular period made by some men to depreciate the importance of the Union? Or why is it suggested that three or four confederacies would be better than one? I am persuaded in my own mind that the people have always thought right on this subject, and that their universal and uniform attachment to the cause of the Union rests on great and weighty reasons, which I shall endeavor to develop and explain in some ensuing papers. They who promote the idea of substituting a number of distinct confederacies in the room of the plan of the convention, seem clearly to foresee that the rejection of it would put the continuance of the Union in the utmost jeopardy. That certainly would be the case, and I sincerely wish that it may be as clearly foreseen by every good citizen, that whenever the dissolution of the Union arrives, America will have reason to exclaim, in the words of the poet: ``\textsc{farewell}! \textsc{a long farewell to all my greatness}."

\vspace{.5cm}
\textsc{Publius}

\vspace{1.5cm}


\chapter[No. 22: The Same Subject Continued (Other Defects of the Present Confederation)]{No. 22\\ {\small The Same Subject Continued (Other Defects of the Present Confederation)}}
To the People of the State of New York:
\vspace{.4cm}

\textsc{In addition }to the defects already enumerated in the existing federal system, there are others of not less importance, which concur in rendering it altogether unfit for the administration of the affairs of the Union.

The want of a power to regulate commerce is by all parties allowed to be of the number. The utility of such a power has been anticipated under the first head of our inquiries; and for this reason, as well as from the universal conviction entertained upon the subject, little need be added in this place. It is indeed evident, on the most superficial view, that there is no object, either as it respects the interests of trade or finance, that more strongly demands a federal superintendence. The want of it has already operated as a bar to the formation of beneficial treaties with foreign powers, and has given occasions of dissatisfaction between the States. No nation acquainted with the nature of our political association would be unwise enough to enter into stipulations with the United States, by which they conceded privileges of any importance to them, while they were apprised that the engagements on the part of the Union might at any moment be violated by its members, and while they found from experience that they might enjoy every advantage they desired in our markets, without granting us any return but such as their momentary convenience might suggest. It is not, therefore, to be wondered at that Mr. Jenkinson, in ushering into the House of Commons a bill for regulating the temporary intercourse between the two countries, should preface its introduction by a declaration that similar provisions in former bills had been found to answer every purpose to the commerce of Great Britain, and that it would be prudent to persist in the plan until it should appear whether the American government was likely or not to acquire greater consistency.\footnote{This, as nearly as I can recollect, was the sense of his speech on introducing the last bill.}

Several States have endeavored, by separate prohibitions, restrictions, and exclusions, to influence the conduct of that kingdom in this particular, but the want of concert, arising from the want of a general authority and from clashing and dissimilar views in the State, has hitherto frustrated every experiment of the kind, and will continue to do so as long as the same obstacles to a uniformity of measures continue to exist.

The interfering and unneighborly regulations of some States, contrary to the true spirit of the Union, have, in different instances, given just cause of umbrage and complaint to others, and it is to be feared that examples of this nature, if not restrained by a national control, would be multiplied and extended till they became not less serious sources of animosity and discord than injurious impediments to the intercourse between the different parts of the Confederacy. ``The commerce of the German empire\footnote{Encyclopedia, article "Empire.``} is in continual trammels from the multiplicity of the duties which the several princes and states exact upon the merchandises passing through their territories, by means of which the fine streams and navigable rivers with which Germany is so happily watered are rendered almost useless." Though the genius of the people of this country might never permit this description to be strictly applicable to us, yet we may reasonably expect, from the gradual conflicts of State regulations, that the citizens of each would at length come to be considered and treated by the others in no better light than that of foreigners and aliens.

The power of raising armies, by the most obvious construction of the articles of the Confederation, is merely a power of making requisitions upon the States for quotas of men. This practice in the course of the late war, was found replete with obstructions to a vigorous and to an economical system of defense. It gave birth to a competition between the States which created a kind of auction for men. In order to furnish the quotas required of them, they outbid each other till bounties grew to an enormous and insupportable size. The hope of a still further increase afforded an inducement to those who were disposed to serve to procrastinate their enlistment, and disinclined them from engaging for any considerable periods. Hence, slow and scanty levies of men, in the most critical emergencies of our affairs; short enlistments at an unparalleled expense; continual fluctuations in the troops, ruinous to their discipline and subjecting the public safety frequently to the perilous crisis of a disbanded army. Hence, also, those oppressive expedients for raising men which were upon several occasions practiced, and which nothing but the enthusiasm of liberty would have induced the people to endure.

This method of raising troops is not more unfriendly to economy and vigor than it is to an equal distribution of the burden. The States near the seat of war, influenced by motives of self-preservation, made efforts to furnish their quotas, which even exceeded their abilities; while those at a distance from danger were, for the most part, as remiss as the others were diligent, in their exertions. The immediate pressure of this inequality was not in this case, as in that of the contributions of money, alleviated by the hope of a final liquidation. The States which did not pay their proportions of money might at least be charged with their deficiencies; but no account could be formed of the deficiencies in the supplies of men. We shall not, however, see much reason to regret the want of this hope, when we consider how little prospect there is, that the most delinquent States will ever be able to make compensation for their pecuniary failures. The system of quotas and requisitions, whether it be applied to men or money, is, in every view, a system of imbecility in the Union, and of inequality and injustice among the members.

The right of equal suffrage among the States is another exceptionable part of the Confederation. Every idea of proportion and every rule of fair representation conspire to condemn a principle, which gives to Rhode Island an equal weight in the scale of power with Massachusetts, or Connecticut, or New York; and to Delaware an equal voice in the national deliberations with Pennsylvania, or Virginia, or North Carolina. Its operation contradicts the fundamental maxim of republican government, which requires that the sense of the majority should prevail. Sophistry may reply, that sovereigns are equal, and that a majority of the votes of the States will be a majority of confederated America. But this kind of logical legerdemain will never counteract the plain suggestions of justice and common-sense. It may happen that this majority of States is a small minority of the people of America;\footnote{New Hampshire, Rhode Island, New Jersey, Delaware, Georgia, South Carolina, and Maryland are a majority of the whole number of the States, but they do not contain one third of the people.} and two thirds of the people of America could not long be persuaded, upon the credit of artificial distinctions and syllogistic subtleties, to submit their interests to the management and disposal of one third. The larger States would after a while revolt from the idea of receiving the law from the smaller. To acquiesce in such a privation of their due importance in the political scale, would be not merely to be insensible to the love of power, but even to sacrifice the desire of equality. It is neither rational to expect the first, nor just to require the last. The smaller States, considering how peculiarly their safety and welfare depend on union, ought readily to renounce a pretension which, if not relinquished, would prove fatal to its duration.

It may be objected to this, that not seven but nine States, or two thirds of the whole number, must consent to the most important resolutions; and it may be thence inferred that nine States would always comprehend a majority of the Union. But this does not obviate the impropriety of an equal vote between States of the most unequal dimensions and populousness; nor is the inference accurate in point of fact; for we can enumerate nine States which contain less than a majority of the people;\footnote{Add New York and Connecticut to the foregoing seven, and they will be less than a majority.} and it is constitutionally possible that these nine may give the vote. Besides, there are matters of considerable moment determinable by a bare majority; and there are others, concerning which doubts have been entertained, which, if interpreted in favor of the sufficiency of a vote of seven States, would extend its operation to interests of the first magnitude. In addition to this, it is to be observed that there is a probability of an increase in the number of States, and no provision for a proportional augmentation of the ratio of votes.

But this is not all: what at first sight may seem a remedy, is, in reality, a poison. To give a minority a negative upon the majority (which is always the case where more than a majority is requisite to a decision), is, in its tendency, to subject the sense of the greater number to that of the lesser. Congress, from the nonattendance of a few States, have been frequently in the situation of a Polish diet, where a single \textsc{vote }has been sufficient to put a stop to all their movements. A sixtieth part of the Union, which is about the proportion of Delaware and Rhode Island, has several times been able to oppose an entire bar to its operations. This is one of those refinements which, in practice, has an effect the reverse of what is expected from it in theory. The necessity of unanimity in public bodies, or of something approaching towards it, has been founded upon a supposition that it would contribute to security. But its real operation is to embarrass the administration, to destroy the energy of the government, and to substitute the pleasure, caprice, or artifices of an insignificant, turbulent, or corrupt junto, to the regular deliberations and decisions of a respectable majority. In those emergencies of a nation, in which the goodness or badness, the weakness or strength of its government, is of the greatest importance, there is commonly a necessity for action. The public business must, in some way or other, go forward. If a pertinacious minority can control the opinion of a majority, respecting the best mode of conducting it, the majority, in order that something may be done, must conform to the views of the minority; and thus the sense of the smaller number will overrule that of the greater, and give a tone to the national proceedings. Hence, tedious delays; continual negotiation and intrigue; contemptible compromises of the public good. And yet, in such a system, it is even happy when such compromises can take place: for upon some occasions things will not admit of accommodation; and then the measures of government must be injuriously suspended, or fatally defeated. It is often, by the impracticability of obtaining the concurrence of the necessary number of votes, kept in a state of inaction. Its situation must always savor of weakness, sometimes border upon anarchy.

It is not difficult to discover, that a principle of this kind gives greater scope to foreign corruption, as well as to domestic faction, than that which permits the sense of the majority to decide; though the contrary of this has been presumed. The mistake has proceeded from not attending with due care to the mischiefs that may be occasioned by obstructing the progress of government at certain critical seasons. When the concurrence of a large number is required by the Constitution to the doing of any national act, we are apt to rest satisfied that all is safe, because nothing improper will be likely \textsc{to be done}, but we forget how much good may be prevented, and how much ill may be produced, by the power of hindering the doing what may be necessary, and of keeping affairs in the same unfavorable posture in which they may happen to stand at particular periods.

Suppose, for instance, we were engaged in a war, in conjunction with one foreign nation, against another. Suppose the necessity of our situation demanded peace, and the interest or ambition of our ally led him to seek the prosecution of the war, with views that might justify us in making separate terms. In such a state of things, this ally of ours would evidently find it much easier, by his bribes and intrigues, to tie up the hands of government from making peace, where two thirds of all the votes were requisite to that object, than where a simple majority would suffice. In the first case, he would have to corrupt a smaller number; in the last, a greater number. Upon the same principle, it would be much easier for a foreign power with which we were at war to perplex our councils and embarrass our exertions. And, in a commercial view, we may be subjected to similar inconveniences. A nation, with which we might have a treaty of commerce, could with much greater facility prevent our forming a connection with her competitor in trade, though such a connection should be ever so beneficial to ourselves.

Evils of this description ought not to be regarded as imaginary. One of the weak sides of republics, among their numerous advantages, is that they afford too easy an inlet to foreign corruption. An hereditary monarch, though often disposed to sacrifice his subjects to his ambition, has so great a personal interest in the government and in the external glory of the nation, that it is not easy for a foreign power to give him an equivalent for what he would sacrifice by treachery to the state. The world has accordingly been witness to few examples of this species of royal prostitution, though there have been abundant specimens of every other kind.

In republics, persons elevated from the mass of the community, by the suffrages of their fellow-citizens, to stations of great pre-eminence and power, may find compensations for betraying their trust, which, to any but minds animated and guided by superior virtue, may appear to exceed the proportion of interest they have in the common stock, and to overbalance the obligations of duty. Hence it is that history furnishes us with so many mortifying examples of the prevalency of foreign corruption in republican governments. How much this contributed to the ruin of the ancient commonwealths has been already delineated. It is well known that the deputies of the United Provinces have, in various instances, been purchased by the emissaries of the neighboring kingdoms. The Earl of Chesterfield (if my memory serves me right), in a letter to his court, intimates that his success in an important negotiation must depend on his obtaining a major's commission for one of those deputies. And in Sweden the parties were alternately bought by France and England in so barefaced and notorious a manner that it excited universal disgust in the nation, and was a principal cause that the most limited monarch in Europe, in a single day, without tumult, violence, or opposition, became one of the most absolute and uncontrolled.

A circumstance which crowns the defects of the Confederation remains yet to be mentioned, the want of a judiciary power. Laws are a dead letter without courts to expound and define their true meaning and operation. The treaties of the United States, to have any force at all, must be considered as part of the law of the land. Their true import, as far as respects individuals, must, like all other laws, be ascertained by judicial determinations. To produce uniformity in these determinations, they ought to be submitted, in the last resort, to one \textsc{supreme tribunal}. And this tribunal ought to be instituted under the same authority which forms the treaties themselves. These ingredients are both indispensable. If there is in each State a court of final jurisdiction, there may be as many different final determinations on the same point as there are courts. There are endless diversities in the opinions of men. We often see not only different courts but the judges of the came court differing from each other. To avoid the confusion which would unavoidably result from the contradictory decisions of a number of independent judicatories, all nations have found it necessary to establish one court paramount to the rest, possessing a general superintendence, and authorized to settle and declare in the last resort a uniform rule of civil justice.

This is the more necessary where the frame of the government is so compounded that the laws of the whole are in danger of being contravened by the laws of the parts. In this case, if the particular tribunals are invested with a right of ultimate jurisdiction, besides the contradictions to be expected from difference of opinion, there will be much to fear from the bias of local views and prejudices, and from the interference of local regulations. As often as such an interference was to happen, there would be reason to apprehend that the provisions of the particular laws might be preferred to those of the general laws; for nothing is more natural to men in office than to look with peculiar deference towards that authority to which they owe their official existence.

The treaties of the United States, under the present Constitution, are liable to the infractions of thirteen different legislatures, and as many different courts of final jurisdiction, acting under the authority of those legislatures. The faith, the reputation, the peace of the whole Union, are thus continually at the mercy of the prejudices, the passions, and the interests of every member of which it is composed. Is it possible that foreign nations can either respect or confide in such a government? Is it possible that the people of America will longer consent to trust their honor, their happiness, their safety, on so precarious a foundation?

In this review of the Confederation, I have confined myself to the exhibition of its most material defects; passing over those imperfections in its details by which even a great part of the power intended to be conferred upon it has been in a great measure rendered abortive. It must be by this time evident to all men of reflection, who can divest themselves of the prepossessions of preconceived opinions, that it is a system so radically vicious and unsound, as to admit not of amendment but by an entire change in its leading features and characters.

The organization of Congress is itself utterly improper for the exercise of those powers which are necessary to be deposited in the Union. A single assembly may be a proper receptacle of those slender, or rather fettered, authorities, which have been heretofore delegated to the federal head; but it would be inconsistent with all the principles of good government, to intrust it with those additional powers which, even the moderate and more rational adversaries of the proposed Constitution admit, ought to reside in the United States. If that plan should not be adopted, and if the necessity of the Union should be able to withstand the ambitious aims of those men who may indulge magnificent schemes of personal aggrandizement from its dissolution, the probability would be, that we should run into the project of conferring supplementary powers upon Congress, as they are now constituted; and either the machine, from the intrinsic feebleness of its structure, will moulder into pieces, in spite of our ill-judged efforts to prop it; or, by successive augmentations of its force an energy, as necessity might prompt, we shall finally accumulate, in a single body, all the most important prerogatives of sovereignty, and thus entail upon our posterity one of the most execrable forms of government that human infatuation ever contrived. Thus, we should create in reality that very tyranny which the adversaries of the new Constitution either are, or affect to be, solicitous to avert.

It has not a little contributed to the infirmities of the existing federal system, that it never had a ratification by the \textsc{people}. Resting on no better foundation than the consent of the several legislatures, it has been exposed to frequent and intricate questions concerning the validity of its powers, and has, in some instances, given birth to the enormous doctrine of a right of legislative repeal. Owing its ratification to the law of a State, it has been contended that the same authority might repeal the law by which it was ratified. However gross a heresy it may be to maintain that a \textsc{party }to a \textsc{compact }has a right to revoke that \textsc{compact}, the doctrine itself has had respectable advocates. The possibility of a question of this nature proves the necessity of laying the foundations of our national government deeper than in the mere sanction of delegated authority. The fabric of American empire ought to rest on the solid basis of \textsc{the consent of the people}. The streams of national power ought to flow immediately from that pure, original fountain of all legitimate authority.

\vspace{.5cm}
\textsc{Publius}

\vspace{1.5cm}

